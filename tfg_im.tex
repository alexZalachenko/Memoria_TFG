\documentclass[a4paper,11pt,titlepage]{scrbook}
\usepackage[utf8]{inputenc}
\usepackage[spanish]{babel}

\usepackage{float} % para usar [H]

% \usepackage[style=list, number=none]{glossary} %si se va a usar glosario, quitar marca de comentario
%\usepackage{titlesec}
%\usepackage{palatino} %usar fot palatino en vez de times roman

%\decimalpoint %revisar
%\usepackage{dcolumn} %revisat
%\newcolumntype{.}{D{.}{\esperiod}{-1}}
%\makeatletter
%\addto\shorthandsspanish{\let\esperiod\es@period@code}
%\makeatother


%\usepackage[chapter]{algorithm}
%\RequirePackage{verbatim}
%\RequirePackage[Glenn]{fncychap}
\usepackage{lscape}
\usepackage{fancyhdr}
\usepackage{graphicx}
\usepackage{afterpage}
\usepackage{longtable}
\usepackage{xcolor}
\definecolor{portada}{RGB}{239,206,53}
\definecolor{base}{RGB}{35,31,32}
\usepackage{pdfpages}
\usepackage{csquotes}
\usepackage[font=scriptsize]{caption}
\usepackage[figuresright]{rotating}
\usepackage{booktabs}
\usepackage{float}
\usepackage{breakcites}
\usepackage[hyphenbreaks]{breakurl}

\newenvironment{itquote}
{\begin{quote}\itshape}
{\end{quote}}

%Instrucciones para poder escribir código y mostrarlo de manera elegante:
\usepackage{listings}
\usepackage{color}

\definecolor{dkgreen}{rgb}{0,0.6,0}
\definecolor{gray}{rgb}{0.5,0.5,0.5}
\definecolor{mauve}{rgb}{0.58,0,0.82}

\lstset{frame=tb,
  language=C,
  aboveskip=3mm,
  belowskip=3mm,
  showstringspaces=false,
  columns=flexible,
  basicstyle={\small\ttfamily},
  numbers=none,
  numberstyle=\tiny\color{gray},
  keywordstyle=\color{blue},
  commentstyle=\color{dkgreen},
  stringstyle=\color{mauve},
  breaklines=true,
  breakatwhitespace=true,
  tabsize=3
}



% minimizar fragmentado de listados
\lstnewenvironment{listing}[1][]
   {\lstset{#1}\pagebreak[0]}{\pagebreak[0]}



% ********************************************************************
% Información sobre el TFG. Comentar lo que NO se desee añadir y sustituir con la información correcta.
% ********************************************************************
\newcommand{\myTitle}{Creación de un videojuego para la enseñanza sobre emprendimiento}
%\newcommand{\mySubtitle}{Subtítulo del proyecto}
\newcommand{\myDegree}{Grado en Ingeniería Multimedia}
\newcommand{\myName}{Álex Verdú Miralles}
\newcommand{\myProf}{Pedro Pernías Peco}
%\newcommand{\myOtherProf}{Nombre Apellido1 Apellido2)}
\newcommand{\myFaculty}{Escuela Politécnica Superior de la Universidad de Alicante}
\newcommand{\myFacultyShort}{EPS UA}
\newcommand{\depTutorOne}{Lenguajes y Sistemas Informáticos}
%\newcommand{\depTutorTwo}{Departamento del cotutor}


\newcommand{\myUni}{\protect{Universidad de Alicante}}
\newcommand{\myLocation}{Alicante}
\newcommand{\myTime}{\today}
%\newcommand{\myVersion}{Version 0.1}

\newcommand{\logoGrado}{imagenes/logoim.jpg}
\newcommand{\logoFacultad}{imagenes/logoeps.jpg}
\newcommand{\logoUniversidad}{imagenes/logoua.jpg}

\usepackage{url}

% Definición de comandos que me son útiles:
%\renewcommand{\indexname}{Índice alfabético}
%\renewcommand{\glossaryname}{Glosario}

\pagestyle{fancy}
\fancyhf{}
\fancyhead[LO]{\leftmark}
\fancyhead[RE]{\rightmark}
\fancyhead[RO,LE]{\textbf{\thepage}}
\renewcommand{\chaptermark}[1]{\markboth{\textbf{#1}}{}}
\renewcommand{\sectionmark}[1]{\markright{\textbf{\thesection. #1}}}


\setlength{\headheight}{1.5\headheight}

\newcommand{\HRule}{\rule{\linewidth}{0.5mm}}
%Definimos los tipos teorema, ejemplo y definición podremos usar estos tipos
%simplemente poniendo \begin{teorema} \end{teorema} ...
\newtheorem{teorema}{Teorema}[chapter]
\newtheorem{ejemplo}{Ejemplo}[chapter]
\newtheorem{definicion}{Definición}[chapter]
 
\newcommand{\bigrule}{\titlerule[0.5mm]}


%Para conseguir que en las páginas en blanco no ponga cabeceras
\makeatletter
\def\clearpage{%
  \ifvmode
    \ifnum \@dbltopnum =\m@ne
      \ifdim \pagetotal <\topskip
        \hbox{}
      \fi
    \fi
  \fi
  \newpage
  \thispagestyle{empty}
  \write\m@ne{}
  \vbox{}
  \penalty -\@Mi
}
\makeatother

\usepackage[pdfborder={000}]{hyperref} %referencia
\hypersetup{
pdfauthor = {\myName (alexverdumiralles (en) gmail (punto) com)},
pdftitle = {\myTitle},
pdfsubject = {},
pdfkeywords = {emprendimiento, lean startup, videojuegos, Unity3D,  programación, desarrollo de videojuegos, C\#},
pdfcreator = {LaTeX con el paquete ....},
pdfproducer = {pdflatex}
}
%AQUI COMIENZA LA LISTA DE FICHEROS A INCLUIR

\begin{document}
\renewcommand{\listtablename}{Índice de tablas} %para sustituir la palabra cuadro por tabla
\renewcommand{\tablename}{Tabla}
\renewcommand{\lstlistingname}{Listado}
\renewcommand{\lstlistlistingname}{Índice de \lstlistingname s}

\frontmatter
\input{portada/portada} %la portada en color
\input{portada/portada_2} %la portada en b/n
\chapter*{Preámbulo}
\thispagestyle{empty}
Los videojuegos han sido parte de mi vida desde que tengo memoria. Empecé jugándolos en una vieja Sega Mega Drive y desde entonces no he dejado de jugarlos en diferentes estilos y plataformas. 

Por otro lado considero que el emprendimiento es una forma muy excitante y noble de alcanzar los retos laborales y profesionales con los que cualquier desarrollador de videojuegos podría soñar.
Es por ello que decidí unir ambos temas y crear un videojuego dedicado a la enseñanza sobre el emprendimiento.

\cleardoublepage %salta a nueva página impar
% Aquí va la dedicatoria si la hubiese. Si no, comentar la(s) linea(s) siguientes
\chapter*{}
\setlength{\leftmargin}{0.5\textwidth}
\setlength{\parsep}{0cm}
\addtolength{\topsep}{0.5cm}
\begin{flushright}
\small\em{
A mi pareja, por su apoyo incondicional. \\
A StackOverflow por estar ahí . \\
Y a toda la comunidad de desarrolladores que \\
ofrece su ayuda en internet de forma desinteresada.
}
\end{flushright}


\cleardoublepage %salta a nueva página impar
% Aquí va la cita célebre si la hubiese. Si no, comentar la(s) linea(s) siguientes
\chapter*{}
\setlength{\leftmargin}{0.5\textwidth}
\setlength{\parsep}{0cm}
\addtolength{\topsep}{0.5cm}
\begin{flushright}
\small\em{
El hombre de negro huía a través del desierto,\\
y el pistolero iba en pos de él
}
\end{flushright}
\begin{flushright}
\small{
Stephen King.
}
\end{flushright}
\cleardoublepage %salta a nueva página impar
 
\tableofcontents
\listoffigures
\listoftables
\lstlistoflistings

\mainmatter %entre frontmatter y mainmatter, la numeración es en romanos.

%a continuación se propone un esquema de trabajo que puede ser alterado justificadamente.
\chapter{Introducción}

El 12 de octubre de 1492 un temerario explorador, Cristobal Colón, y su tripulación pisan la arena de una isla muy al oeste de Europa conocida como Guanahani. Este hecho marca un hito en la historia de la humanidad pues los cambios culturales, económicos, políticos y militares que produce dan lugar a la llamada Edad Moderna.\\
Colón vio una oportunidad de negocio en el control de las rutas comerciales que unían Europa con Asia pues eran recorridas por miles de comerciantes que traían especias y productos de lujo desde las tierras de Extremo Oriente. El comercio además se realizaba por tierra, lo que lo convertía en un proceso lento, inseguro e ineficiente, además de enriquecedor para los árabes que controlaban las rutas comerciales.

El proyecto tenía un gran interés económico pues como se ha dicho anteriormente, el control de una ruta comercial con Asia era muy lucrativo, pero a su vez tenía un gran riesgo ya que el futuro de la expedición era tremendamente incierto y había pocas posibilidades de encomendarse al vasto oceano y volver para contarlo. Debido a esta incertidumbre sobre el retorno de la inversión a Colón le fue complicado encontrar financiación para su proyecto, hasta que finalmente, tras recurrir a varios monarcas y mecenas,  los Reyes Católicos le proveyeron de los rescursos necesarios para iniciar su aventura.

Se podría considerar a Cristobal Colón como un emprendedor, a pesar de que el término fue usado por primera vez doscientos años después por el economista Richard Cantillon que define al emprendedor como ''La persona que paga un cierto precio para revender un producto a un precio incierto, por ende tomando decisiones acerca de la obtención y el uso de recursos, y admitiendo consecuentemente el riesgo en el emprendimiento" \cite[pág 21]{ashokbhanudasnavale2013}.
De esta definición se puede apreciar que un emprendedor inicia proyectos y acepta la incertidumbre y el riesgo que ello conlleva, puesto que en caso de desastre es él quien carga con la responsabilidad.

La actitud emprendedora ha sido una constante a lo largo de la historia de la humanidad: desde Cristobal Colón hasta Bill Gates, pasando por Leonardo Da Vinci, Henry Ford o Nikola Tesla; hombres y mujeres con coraje han empezado proyectos bajo una idea prometedora y asumiendo grandes riesgos, motivados por la pasión y las perspectivas de éxito. 

El emprendimiento es una actividad especialmente necesaria para el progreso de una sociedad pues es un proceso que crea riqueza, innovación y empleo. Los emprendedores crean productos y servicios revolucionarios que hacen la vida de las personas más fácil, mejorando por ello su calidad de vida. Además suele ser una salida muy recurrida en épocas de crisis económicas debido a la escasez de empleo.

\section{Emprendimiento y el fenómeno startup}

Cada vez es más frecuente escuchar el término startup, pequeñas empresas dedicadas al ámbito tecnológico que alcanzan en pocos años grandes cuotas de mercado y se venden por millones de euros a empresas más grandes.

 El fenómeno goza de tanta popularidad que ha inspirado incluso a series como Silicon Valley, que narra las aventuras de un grupo de jóvenes ingenieros que crean una startup tecnológica y se enfrentan al reto de sobrevivir en un ecosistema hostil como es el mercado; la película Piratas de Silicon Valley, que narra la historia de enfrentamiento entre Microsoft y Apple; la película La red social que cuenta la historia de Mark Zuckerberg y como crea la red social Facebook.

Llegado a este punto cabe preguntarse: ¿Qué es exactamente una startup?. Es un error común pensar que las startup son simplemente versiones más pequeñas de empresas grandes. En palabras de los gurús del emprendimiento Steve Blank y Bob Dorf, ''Una startup es una organización temporal en busca de un modelo de negocio rentable, que pueda repetirse y que es escalable"\cite{steveblankbobdorf2013}.\\
De la anterior definición se puede extraer que una startup:
\begin{itemize}
	\item Es una organización temporal, es decir, el objetivo no es ser siempre una startup. El objetivo es convertirse en una empresa consolidada.
	\item No conoce con seguridad cual va a ser su actividad. En su lugar parten de un modelo de negocio temporal que va evolucionando a medida que interactúa con el mercado.
	\item Busca un modelo de negocio repetible y escalable, que le permita ejecutar dicho modelo de negocio durante un tiempo indefinido y además expandirse.
\end{itemize}
El emprendimiento es inherente al fenómeno startup pues la incertidumbre es un pilar fundamental al crear una de estas empresas, que ni siquiera tienen un modelo de negocio que se pueda asegurar que va a funcionar.

\section{Estado actual del emprendimiento en España}

Si bien el fenómeno startup nació en EEUU y es allí donde está más consolidado, en España es una tendencia igualmente extendida. Atendiendo a cifras de financiación ''en 2015, las startups españolas lograron financiación por valor de 500 millones de euros, un 87\%   más que en 2014, cuando apenas se invirtieron 286 millones de euros" \cite{albertoiglesiasfraga2016}.\\
Actualmente en nuestro país hay 1783 empresas emergentes distribuidas principalmente en Madrid, Cataluña y la Comunidad valenciana. Dichas empresas se dedican principalmente al ecommerce(22\%), social media(13\%) y las empresas(12\%). En cuanto a la financiación, 172 inversores operan en el ámbito startup a lo largo de la península \cite{startupxplore2017} y los fondos que han aportado crecen año a año: ''en 2013 tres startups lograron rondas de financiación que superaran los 10 millones de euros [...] en 2014, esta cifra aumentó a cuatro [...] el pasado año la explosión no tuvo parangón, ya que hasta 13 startups lograron capitalizar más de 10 millones de euros para fomentar su desarrollo" \cite{albertoiglesiasfraga2016}. 

\section{Lean startup}

Lean startup es un modelo de gestión empresarial dinámico ampliamente utilizado en la creación de empresas emergentes. En contraposición a las metodologías tradicionales, Lean startup se basa en ciclos de desarrollo cortos que permiten sacar el producto al mercado de forma temprana. De este modo se puede obtener retroalimentación de los clientes en las etapas iniciales de la empresa, lo que da lugar a que el producto cambia y se adapta a las necesidades de los clientes.

El primer paso para crear una startup según esta metodología es plasmar las hipótesis sobre el modelo de negocio en el Lean canvas \ref{leanCanvas}( como se cita la imagen????? http://innokabi.com/wp-content/uploads/2013/09/lienzo-lean-canvas-de-ash-maurya.jpg) , basado en el modelo de negocio de Alexander Osterwalder.\\
Estas hipótesis no conforman el modelo de negocio definivo, si no que irán evolucionando a lo largo de la vida de la empresa de acuerdo al feedback de los clientes. Esta evolución del producto en relación a los deseos del clientes se denomina \textbf{customer development} y es uno de los conceptos claves en Lean startup.

El ciclo de vida de una startup se basa en tres pasos fundamentales que se repiten cíclicamente:
\begin{itemize}
	\item Construir: se diseña el producto en función de las hipótesis que se establecen en el \textbf{lean canvas}. En la primera iteración se crea una versión del producto que tenga las minímas funcionalidades necesarias para aportar valor a los potenciales clientes. Esta versión del producto se denomina \textbf{producto minímo viable}. El objetivo de esta etapa es ''comenzar a recopilar datos y medir resultados. Este modelo de producto no busca ser el resultado final sino un producto suficiente para testar la reacción del potencial cliente" \cite{antevenio2016}.
	\item Medir: tras contrastar nuestras hipótesis de negocio con los clientes a través del \textbf{producto minímo viable} obtenemos información sobre nuestro producto y sobre la propia empresa mediante \textbf{métricas clave}. Dichas métricas (tales como ¿Cuánto cuesta captar un cliente? o ¿Cuánto dinero gastamos mensualmente?) son valoraciones objetivas sobre el rendimiento del producto y de la empresa, y calcularlas de forma periódica es importante ya que permite trazar una evolución y detectar errores y mejoras en la estrategia empresarial.
	\item Aprender: es una etapa clave ya que si el conocimiento obtenido se aplica, se estará más cerca de crear un producto que los clientes quieran comprar. ''este conocimiento adquirido se debe aplicar a un nuevo proceso que comienza de nuevo. Se vuelve a crear un producto, que será una mejora del mismo lo que hace arrancar de nuevo el círculo de crear, medir y aprender" \cite{antevenio2016}. Al llegar a este punto las startups se deben plantear si realizar un pequeño ajuste al producto y volver a \textbf{iterar} o si bien, en caso de que los resultados del producto hayan sido un desastre, hacer cambios de base al modelo de negocio. Estos cambios que afectan a una o más hipótesis del \textbf{lean canvas} se denominan \textbf{pivotar} y consisten en ''cambiar una hipótesis fundamental sobre el producto, la estrategia, y el motor de crecimiento" \cite{emooc}.
\end{itemize}

El proceso se puede realizar cuantas veces sea necesario hasta conseguir el producto que se considere más acorde al cliente. La metodología Lean Startup no trata de evitar que fallemos en el primer intento de lanzar al mercado nuestro servicio, sino que trata de que ese fallo nos salga más ‘barato’ al haber empleado una cantidad considerablemente menor de tiempo y de recursos materiales y económicos \cite{andreapelaez} .

\begin{figure}
\begin{center}
\includegraphics[scale=0.6]{imagenes/leanCanvas.jpg}
\caption{Lean canvas}
\label{leanCanvas}
\end{center}
\end{figure}





\chapter{Marco Teórico: emprendimiento y videojuegos}
\label{marcoteorico}

El tema del emprendimiento no solo ha proporcionado ideas para la creación de películas cinematográficas o series de televisión. Como apoyo a los cursos formativos sobre emprendimiento han surgido multitud de juegos para enseñar y aplicar de forma práctica conceptos sobre la creación de iniciativas empresariales.

Estos juegos son una parte importante de la formación ya que permiten aprender de forma más amena. Además el utilizar de forma práctica los conocimientos adquiridos ayuda a que se entiendan mejor y se recuerden durante más tiempo.

Para los juegos mencionados anteriormente existen varios formatos que serán explicados en detalle a continuación. 

\section{Actividades en grupo}

Es el tipo más sencillo y tradicional. Solo necesita de los participantes y una actividad previamente elegida. Una de las ventajas que tiene este tipo de actividades es que ponen a las personas en contacto directo, de modo que tienen que dejar a un lado la vergüenza e interactuar como lo harían ante clientes, inversores o trabajadores. Además estos encuentros pueden servir para hacer contactos útiles en un futuro.

Este tipo de actividades potencian las habilidades sociales y la creatividad de los participantes ya que los únicos elementos del juego son las personas, y las mecánicas del juego son sus discursos, explicaciones, gestos, actuaciones, etc.

\section{Juegos de mesa}

Los juegos de mesa mantienen muchos de los aspectos positivos de las actividades en grupo ya que también son presenciales, con las ventajas y desventajas que ello conlleva.

Además estos juegos pueden ser más divertidos debido a que introducen elementos como tableros, cartas, textos o ilustraciones entre otros elementos. Son especialmente interesantes para personas que no se sientan cómodas con las actividades en grupo debido al alto grado de interacción que demandan.

También es más fácil el uso de mecánicas complejas ya que hay instrumentos para contabilizar y describir el estado del juego: dados para contar el número de vidas, poder mágico o turnos; tableros con diferentes casillas, territorios o zonas; fichas de jugador con parámetros, habilidades, y características.

Actualmente hay numerosos juegos de mesa entre los que destacaremos:
\begin{description}
\item[Colonos de Catán: ] \footnote{\url{http://devir.es/producto/catan/}} (figura \ref{catan})  Apto para 6 jugadores (a través de una expansión), este juego de gestión de recursos y comercio nos pone en la piel de un colono que debe ir construyendo sus aldeas y caminos. En Colonos de Catán prima tu habilidad para negociar por el contrario y tu capacidad de estrategia a medio y largo plazo. \cite{faceentrepreneurship2016}

El Catán [...] evita el enfrentamiento tan directo y obliga a negociar para ganar.

Ahí reside una de las claves, en el hecho que de entrada nadie posea recursos suficientes de todos los tipos para progresar. En cada turno se comercia con las materias primas, un trueque básico que permite saber cómo funciona un mercado libre en el que cada uno tiene sus propios intereses. \cite{albertini2015}

Lleva unas 18 millones de copias vendidas [...] Ha aparecido en \textquote{The Big Bang Theory} [...] Mark Zuckerberg se ha declarado adicto, \textquote{es uno de esos juegos a los que debes jugar si no quieres ser el \textquote{margi} de Silicon Valley} \cite{albertini2015}

\begin{figure}
\begin{center}
\includegraphics[scale=0.3]{imagenes/catan.jpg}
\caption{Tablero del juego Colonos de Catán.  Fuente: Blog del club ERTAI. \url{https://clubertai.wordpress.com/2013/03/10/ludoteca-del-club-los-colonos-de-catan/}}
\label{catan}
\end{center}
\end{figure}


\item[Pandemia: ] \footnote{\url{http://zacatrus.es/pandemia.html}} (figura \ref{pandemia}) En Pandemia somos un grupo de hasta 4 científicos que tienen que mantener a raya una serie de virus, o de lo contrario la Humanidad tendrá un grave problema. [...] debes aprender a formar equipo y hacerlo funcionar si quieres tener éxito en tu \textquote{startup}. En este sentido, Pandemia puede ser la terapia perfecta para ti y tu equipo ya que o colaboráis y funcionáis perfectamente engrasados o correréis el riesgo de fracasar. Y también en la vida real. \cite{faceentrepreneurship2016}

\begin{figure}
\begin{center}
\includegraphics[scale=0.25]{imagenes/pandemia.jpg}
\caption{Tablero del juego Pandemia.  Fuente: Blog Estantería de juegos. \url{http://estanteriadejuegos.blogspot.com.es/2013/12/resena-pandemia.html}}
\label{pandemia}
\end{center}
\end{figure}

\item[Flea market:] \footnote{\url{https://boardgamegeek.com/boardgame/172410/flea-market}} (figura \ref{fleaMarket}) Juego de dados en el que tendrás que descubrir los tesoros escondidos en un mercado de segunda mano. Tú serás el cliente que compra, y tu objetivo es adquirir bienes lo más barato posible para venderlos por más dinero una vez el dado de la demanda dice que ya son populares de nuevo. \cite{mariagonzalez2015}

\begin{figure}
\begin{center}
\includegraphics[scale=0.4]{imagenes/fleaMarket.jpg}
\caption{Tablero del juego Flea market.  Fuente: Tienda online Shopyourway. \url{http://www.shopyourway.com/articles/355984}}
\label{fleaMarket}
\end{center}
\end{figure}
\end{description}

\section{Videojuegos}

Con el creciente éxito del mercado de los videojuegos y el entretenimiento digital, los videojuegos sobre el emprendimiento no se han hecho esperar. Los hay de diferentes tipos y temáticas aunque todos ellos comparten la esencia de dirigir un negocio.

\begin{description}
         \item[Gestionar un negocio:] en este tipo de videojuegos el jugador es el gerente de un negocio como por ejemplo una cafetería o una peluquería. El objetivo es gestionar la actividad diaria del establecimiento y ampliar el mismo utilizando los ingresos obtenidos. Ejemplos de este tipo de videojuegos son \textquote{Diner dash\footnote{\url{https://es.wikipedia.org/wiki/Diner_Dash}}} o \textquote{My Cafe: Recipes \& Stories\footnote{\url{https://play.google.com/store/apps/details?id=com.melesta.coffeeshop\&hl=es}}}. Este tipo de juegos tienen un carácter más infantil y recreativo y distan de ser un juego basado realmente en el emprendimiento y el mundo \textquote{startup}. El motivo es que el peso del juego reside principalmente en la gestión cotidiana del negocio como servir pedidos, cobrar a los clientes o comprar materiales; valores muy importantes para el emprendedor como la creatividad, la estrategia, la pasión o la innovación no suelen tener cabida en estos juegos.
         

	\item[Hipster CEO:] \footnote{\url{http://www.hipsterceo.com/}} (figura \ref{hipster01}) este juego se puede considerar la contraposición al juego mencionado en el punto anterior ya que elimina las tareas de interacción directa con el cliente y en su lugar el jugador se centra en la gestión desde un punto de vista más técnico. El objetivo de este juego es lanzar y dirigir una \textquote{startup} con todo lo que ello conlleva.
	
Entre las tareas del jugador se pueden destacar: contratar personal, dedicar recursos a los diferentes departamentos (ventas, producción o marketing) 

Para ello se dispone de un dashboard con varias \textbf{métricas clave} sobre la \textquote{startup} que ayudan al jugador a tomar decisiones acerca del rumbo de la misma. El juego no dispone de ningún tipo de historia o personajes que controlar, si no que la interacción con el mismo es a través de informes, dashboards, emails, etc.

\begin{figure}
\begin{center}
\includegraphics[scale=0.6]{imagenes/hipster01.jpg}
\caption{Captura de pantalla de iPhone con el juego Hipster CEO}
\label{hipster01}
\end{center}
\end{figure}

	\item [U-startup:] \footnote{\url{http://ustartup.es/}} (figura \ref{ustartup01}) es un juego desarrollado por la Universidad de Cádiz en España, en colaboración con la Cátedra de Emprendedores y OmniumLab. Lo novedoso de esto es que es el primer videojuego sobre emprendimiento con el que aprenderás a construir tu negocio con la metodología de CANVAS. \cite{danieladelbosque2016}
	
Este juego se puede considerar la contraposición a los juegos mencionados en el punto anterior ya que explícitamente se muestran elementos típicos de la metodología \textbf{lean startup} tales como el \textbf{lean canvas}. El juego se basa en completar este canvas con las hipótesis de negocio del jugador a la vez que se completa una aventura en la que un fantasma es el mentor del jugador en el mundo empresarial, la pitonisa ayuda a encontrar los clientes objetivo y otros personajes y escenarios intervienen.
	
\begin{figure}
\begin{center}
\includegraphics[scale=0.6]{imagenes/ustartup01.jpg}
\caption{Vista del lean canvas en el juego U-startup}
\label{ustartup01}
\end{center}
\end{figure}

\end{description}


\section{Motores de videojuegos}

La creación de un videojuego sin la utilización de herramientas que agilicen el proceso puede ser una tarea más que ardua. Son muchos los motores de videojuegos que hacen este proceso más sencillo proporcionando utilidades como sistema de renderizado, motor físico o gestor de sonido. Además estos motores proporcionan código encargado de la gestión de la escena, de la memoria, de las entidades, etc.

Actualmente existen una gran cantidad de motores de videojuegos. Es por ello que no se nombrarán todos ya que quedaría fuera del ámbito de este trabajo. A continuación se ofrecerá una breve explicación de los motores de videojuegos más notorios actualmente: Unity3D y Unreal engine 4

\section{Unity3D}

Unity3D es uno de los motores más reconocidos a día de hoy. Gran parte de su buen nombre deriva de que se pueden crear videojuegos de forma sencilla sin necesitar demasiados conocimientos sobre desarrollo. A su vez ofrece todas las herramientas necesarias para que los profesionales no se vean limitados y puedan alcanzar todo su potencia. Muestra de ello es la creación de títulos de gran calidad tales como The long dark \footnote{http://hinterlandgames.com/the-long-dark/}. Unity3d suele ser el motor por excelecia cuando se habla de desarrollo de videojuegos para móviles, pero como se ha mostrado con el título anterior no se limita a dicha plataforma. En su lugar Unity3D ofrece la posibilidad de exportar un proyecto hecho con este motor a multitud de plataformas sin tener que cambiar una sola línea de código. Entre las plataformas soportadas se incluyen IOS, Android, Windows, Linux o Mac entre otras.

\section{Unreal engine 4}

Unreal engine 4 es un motor creado por la compañía Epic games\footnote{https://www.epicgames.com}, autora de famosos títulos tales como Paragon\footnote{https://www.epicgames.com/paragon} o Unreal tournament\footnote{https://www.epicgames.com/unrealtournament/}. 
Unreal engine 4 es la evolución del éxitoso motor Unreal development kit (UDK) creado por la misma compañía.


Las características que hacen a Unreal engine 4 único son su potente motor gráfico, que hace que en la actualidad sea la elección más común a la hora de desarrollar videojuegos para consolas. Otro punto interesante de Unreal engine 4 es el sistema de scripting basado en blueprints. Esto significa que se pueden definir comportamientos para entidades del videojuego prácticamente sin necesidad de escribir código. ''La programación'' se realiza de forma gráfica arrastrando y seleccionando componentes como se puede ver en la siguiente imagen (ver fig. \ref{blueprints}).

\begin{figure}
\begin{center}
\includegraphics[scale=0.4]{imagenes/blueprints.png}
\caption{Uso de blueprints en Unreal engine 4. Fuente: www.gamedev.es}
\label{blueprints}
\end{center}
\end{figure}

Alternativamente se puede optar por la programación tradicional, que se realizará usando el lenguaje C++. De hecho se recomienda y se incentiva el uso de ambas técnicas de forma simultánea para lograr los mejores resultados.

\section{Propuesta de desarrollo}

Como se deduce del anterior análisis, el mundo del emprendimiento es un negocio en auge. Y evidentemente el mundo de la educación asociada al emprendimiento también lo es, pues no existiría el uno sin el otro.

Frente a los métodos tradicionales de enseñaza se propone la creción de un videojuego que sin comprometer la calidad de los conocimientos adquiridos por el jugador, ofrezca una experiencia interactiva y divertida.

El videojuego será una aventura gráfica para dispositivos móviles que durante el transcurso de la misma ofrecerá lecciones sobre emprendimiento y Lean startup a través de diálogos con los diferentes personajes del juego.
\chapter{Objetivos}

\section{Objetivo principal}

El objetivo principal de este proyecto la creación un videojuego mediante el cual se puedan aprender conceptos claves acerca del emprendimiento y en concreto sobre la metodología \textquote{Lean startup}.


\section{Objetivos específicos}

\begin{enumerate}
	\item Implementar un sistema conversacional en el que el jugador pueda seleccionar las respuestas que desea dar y en función de ello cambie la historia.
	\item Emplear el concepto \textquote{storytelling} para diseñar y escribir una historia que se desarrolle a lo largo de conversaciones entre el jugador y los demás personajes.
	\item Crear una interfaz de usuario que cumpla con unos criterios de calidad y que permita al jugador interactuar con la aplicación.
	\item Crear modelos 3D de personajes y otros elementos tales como edificios. Texturizar dichos modelos.
\end{enumerate}


\section{Objetivos secundarios}

\begin{enumerate}
	\item Planificar el desarrollo de un proyecto de desarrollo de software y llevarlo a cabo.
	\item Crear un GDD (documento de diseño de videojuego) donde se documente con precisión como será el juego.
	\item Aprender nuevas habilidades sobre el motor de videojuegos \textquote{Unity3D} y el lenguaje de programación C\#.
	\item Aprender nuevas habilidades sobre el software de modelado 3D \textquote{Blender}\footnote{https://www.blender.org/}.
\end{enumerate}




\chapter{Metodología}

Para la realización del proyecto, la creación del videojuego, se dispondrán de 6 meses que se dividirán de la siguiente forma: inicialmente un mes y medio para la investigación, especificación del producto y planificación; cuatro meses desarrollando el producto; finalmente medio mes para dar los últimos retoques al juego y lanzar una versión beta.

Para el desarrollo del videojuego se utilizará una metodología ágil, ya que al disponer de poco tiempo para el desarrollo se necesitará tener un producto cuanto antes para poder enfrentarlo al público y obtener el feedback de este. Utilizando una metodología ágil se focaliza el esfuerzo en el desarrollo en lugar de en una excesiva planificación. De esta forma 
\begin{itquote}
	se trabaja realizando entregas parciales pero funcionales del producto. De ese modo, es posible entregar en el menor intervalo de tiempo posible una versión funcional del producto.
	\begin{flushright}
	 	\cite{eduardomartinez2014}.
 	\end{flushright}
\end{itquote}

Además la utilización de estas metodologías ayuda enormemente a reducir el riesgo: al crear un videojuego, en este caso educativo, a pesar de las mejores intenciones de los desarrolladores no se puede saber con certeza si será del agrado de los jugadores.

Es por ello que la mejor estrategia posible es crear un producto minimo viable (MVP) y que posteriormente se desarrolle y corrija según los deseos de aquellos que lo jugarán. 

Existen actualmente una gran cantidad de metodologías ágiles. Entre las más populares se pueden destacar: 

\begin{itemize}
	\item Scrum: es una metodología orientada a equipos. Proporciona herramientas para el seguimiento diario del proyecto, la planificación de trabajo de forma iterativa y la comunicación y cooperación de los integrantes del grupo.
	\item Extreme Programming: orientada a equipos con pocos programadores. 
		\begin{itquote}
			se aplica en equipos con muy pocos programadores quienes llevan muy pocos procesos en paralelo. Consiste entonces en diseñar, implementar y programar lo más rápido posible, hasta en casos se recomienda saltar la documentación y los procedimientos tradicionales.
			\begin{flushright}
	 			\cite{opheliapastrana2015}.
 			\end{flushright}
		\end{itquote}
	\item Kanban: esta metodología propone dividir el trabajo en diferentes etapas bien diferenciadas. El objetivo es evitar los cuellos de botella limitando el trabajo en curso. Para ello, se establece un límite de trabajo en curso, lo que obliga a que cuando una tarea se empieza se debe terminar antes de iniciar una nueva.
\end{itemize}

La metodología ágil a utilizar será Kanban ya que es tremendamente sencilla de implementar: con unas simples tarjetas se pueden especificar las tareas a realizar, y con un tablero se pueden crear columnas que representan los estados de las diferentes tareas.

Dada la facilidad con la que se puede implementar Kanban, y que no es un sistema directamente orientado a equipos como SCRUM, será muy adecuado para el proyecto.


En cuanto al producto a desarrollar, los cuatro meses de desarrollo se dividirán en iteraciones de duración variable. Al finalizar la segunda iteración se espera tener un mínimo producto viable (MVP) y en las dos siguientes se perfeccionará dicho producto.

En cuanto al tiempo de desarrollo se dividirá en cuatro iteraciones: 

\begin{itemize}
	\item Primera iteración: desarrollo de la parte software del MVP usando placeholders en lugar de los modelos 3D y los demás elementos gráficos
	\item Segunda iteración: inclusión de los modelos 3D e imágenes definitivas
	\item Tercera iteración: recolección de feedback y corrección de errores. Optimización del juego
	\item Cuarta iteración: recolección de feedback y corrección de errores. Detalles finales del juego
\end{itemize}

Para ilustrar esta distribución de tareas se ha creado un diagrama de Gantt simplificado como se puede apreciar en la siguiente imagen \ref{gantt01}.

\begin{figure}
	\begin{center}
		\includegraphics[scale=0.6]{imagenes/GanttDiagram.jpg}
		\caption{Diagrama de Gantt con la distribución temporal y las dependencias de las tareas}
		\label{gantt01}
	\end{center}
\end{figure}

Se utilizará la herramienta online TargetProcess \footnote{www.targetprocess.com} para disponer de un tablero virtual \textquote{Kanban} en el cual poner las tarjetas y separarlas por procesos. En este aspecto goza de más popularidad la herramienta Trello \footnote{https://trello.com/}, aunque TargetProcess es más completa y ofrece muchas opciones como filtros, métricas y otras utilidades para monitorizar el trabajo.

%\section{Producción de contenido}
%\subsection{Modelos 3D}
%Los modelos 3D a utilizar se obtendrán de sitios web que ofrezcan este tipo de recursos de forma gratuita y con licencia que permita su uso comercial. En caso de que alguno de los modelos no se encontrara se modelaría el mismo utilizando el software Blender. 
%\subsection{Música y sonidos}
%qweqwe
%\subsecion{Motor de videojuego}
%asdasda



\chapter{Desarrollo}

\section{Descripción general}

Para describir el desarrollo se ha decidido seguir el estándar IEEE 830 para la especificación de requisitos \footnote{\url{www.fdi.ucm.es/profesor/Gmendez/docs/is0809/ieee830.pdf}}.

\section{Perspectiva del producto}

El sistema a construir consistirá en una aplicación móvil para dispositivos Android \footnote{\url{www.android.com}} desarrollada con el motor de videojuegos Unity3D \footnote{\url{https://unity3d.com/es/}}. 

La aplicación no formará parte de un sistema mayor, será un videojuego totalmente independiente. Aun así se hará uso de servicios de terceros tales como librerías de software, frameworks y APIs entre otros.

\section{Funcionalidad del producto}

El videojuego deberá permitir el movimiento del jugador en el plano 2D, además podrá interactuar con los NPCs y dialogar con ellos. Se podrán completar objetivos y logros para de este modo avanzar en la historia.

El juego además enseñará conceptos clave sobre \textquote{Lean Startup} y sobre el emprendimiento en general.

\section{Características de los usuarios}

El perfil de un consumidor de formación sobre emprendimiento es muy amplio: desde jóvenes recién graduados llenos de optimismo hasta personas de mediana edad que desean reinventarse y dejar de ser asalariados. 

Es por ello que es difícil concretar un perfil ya que son diferentes personas de diferentes edades y perfiles socioculturales las que desean aventurarse en el emprendimiento.

En cualquier caso, sí se pueden encontrar aspectos comunes en esta gran variaded de usuarios:

\begin{itemize}

\item conocimiento tecnológico y cómo desenvolverse con aplicaciones móviles
\item interés por el mundo del emprendimiento y la empresa
\item interés por los videojuegos
\item interés por los videojuegos educativos

\end{itemize}

\section{Restricciones}

Existen varias limitaciones a tener en cuenta a la hora de diseñar y desarrollar el sistema:

\begin{itemize}

\item para la realización del proyecto en versión MVP se dispondrá de un presupuesto cero. Es por ello que las herramientas, frameworks y demás productos que se utilicen deberán ser gratuitas.
\item el motor de videojuegos a utilizar deberá ser Unity 3D ya que se tiene conocimiento del mismo y aprender a utilizar otro motor está fuera del alcance de este trabajo.
\item el lenguaje de programación a utilizar será C\# ya que de entre los disponibles para Unity 3D es el más adecuado por potencia, documentación y dominio por parte del desarrollador.
\item el sistema operativo objetivo será Android ya junto con los sistemas operativos para ordenador no requiere ninguna licencia para publicar aplicaciones. La plataforma serán los dispositivos móviles ya que es muy sencillo llegar al público de esta plataforma, aunque no tanto hacerse hueco entre dicha audiencia.

\end{itemize}

\section{Requisitos específicos}

\subsection{Interfaces de usuario}

Respecto a las interfaces de usuario se pueden observar dos estilos claramente diferenciados: las interfaces del menú principal y las de la pantalla de juego. En cuanto a las primeras deberán seguir un estilo minimalista y utilizar controles simples; respecto de las segundas, la simplicidad es obligatoria.

Es un requisito imprescindible que la interfaz mostrada durante el juego no sea intrusiva y entorpezca la experiencia de usuario. Esto se conseguirá disponiendo pequeños botones en la pantalla situados de forma estratégica para que la visión del jugador se centre principalmente en el mundo del juego y los personajes.

Un requisito común de las interfaces de usuario es que debido a que serán mostradas en un dispositivo móvil tendrán que adecuarse a una pantalla pequeña y recibir la interacción del usuario mediante toques en la pantalla del dispositivo.

\subsection{Requisitos funcionales}

Para describir las funcionalidades del sistema se utilizará una aproximación basada en mecánicas. Cada posible acción del usuario sobre el sistema se considerará una mecánica y se definirá como un requisito. 

Dichas mecánicas son activadas al detectarse cierto estímulo. Por ejemplo al producir el estímulo de tocar en algun lugar del mundo, se desencadena la mecánica de movimiento.

Al identificador de cada funcionalidad le acompañarán unas siglas (USR, UI, SYS) dependiendo de si dicha funcionalidad se refiere a mecánicas del usuario, la interfaz de usuario o al sistema.

\subsection{Requisitos de usuario}
\label{requisitosUsuario}

\begin{table}[H]
\centering
\label{my-label}
\begin{tabular}{|l|l|}
\hline
\textbf{Identificador} & RF-USR-01                                                                                                                                                              \\ \hline
\textbf{Nombre}        & Mover personaje                                                                                                                                                        \\ \hline
\textbf{Requerimiento} & El usuario podrá elegir donde mover al personaje controlado                                                                                                            \\ \hline
\textbf{Descripción}   & \begin{tabular}[c]{@{}l@{}}Al tocar con el dedo en cualquier punto de la pantalla el personaje\\   controlado se moverá a esa posición (solo en el eje x)\end{tabular} \\ \hline
\textbf{Prioridad}     & Imprescindible                                                                                                                                                         \\ \hline
\end{tabular}
\end{table}

\begin{table}[H]
\centering
\label{my-label}
\begin{tabular}{|l|l|}
\hline
\textbf{Identificador} & RF-USR-02                                                                                                                                           \\ \hline
\textbf{Nombre}        & Interactuar con NPCs                                                                                                                                \\ \hline
\textbf{Requerimiento} & El usuario podrá seleccionar un NPC con el que interactuar                                                                                          \\ \hline
\textbf{Descripción}   & \begin{tabular}[c]{@{}l@{}}Al tocar con el dedo sobre un NPC, si se está lo suficientemente cerca se\\   abrirá el menú conversacional\end{tabular} \\ \hline
\textbf{Prioridad}     & Imprescindible                                                                                                                                      \\ \hline
\end{tabular}
\end{table}

\begin{table}[H]
\centering
\label{my-label}
\begin{tabular}{|l|l|}
\hline
\textbf{Identificador} & RF-USR-03                                                                                                                                      \\ \hline
\textbf{Nombre}        & Cambiar de escenario                                                                                                                           \\ \hline
\textbf{Requerimiento} & Se podrá navegar entre escenarios                                                                                                              \\ \hline
\textbf{Descripción}   & \begin{tabular}[c]{@{}l@{}}Al clicar en una de las puertas de cada escenario se avanzará al\\   escenario asociado a dicha puerta\end{tabular} \\ \hline
\textbf{Prioridad}     & Imprescindible                                                                                                                                 \\ \hline
\end{tabular}
\end{table}

\subsection{Requisitos de interfaz}

\begin{table}[H]
\centering
\label{my-label}
\begin{tabular}{|l|l|}
\hline
\textbf{Identificador} & RF-UI-01                                                                                                                                                \\ \hline
\textbf{Nombre}        & Navegar menú principal                                                                                                                                  \\ \hline
\textbf{Requerimiento} & Se podrá cambiar entre las diferentes pantallas del menú principal                                                                                      \\ \hline
\textbf{Descripción}   & \begin{tabular}[c]{@{}l@{}}Arrastrando con el dedo en la pantalla hacia la derecha/izquierda se\\   cambiará a la correspondiente pantalla\end{tabular} \\ \hline
\textbf{Prioridad}     & Imprescindible                                                                                                                                          \\ \hline
\end{tabular}
\end{table}

\begin{table}[H]
\centering
\label{my-label}
\begin{tabular}{|l|l|}
\hline
\textbf{Identificador} & RF-UI-02                                                                                                                                        \\ \hline
\textbf{Nombre}        & Comenzar juego                                                                                                                                  \\ \hline
\textbf{Requerimiento} & \begin{tabular}[c]{@{}l@{}}Se empezará el juego al seleccionar el botón correspondiente en el menú\\   principal\end{tabular}                   \\ \hline
\textbf{Descripción}   & \begin{tabular}[c]{@{}l@{}}En el el menú principal en la vista inicial se podrá comenzar el juego al\\   presionar el botón "play"\end{tabular} \\ \hline
\textbf{Prioridad}     & Imprescindible                                                                                                                                  \\ \hline
\end{tabular}
\end{table}

\begin{table}[H]
\centering
\label{my-label}
\begin{tabular}{|l|l|}
\hline
\textbf{Identificador} & RF-UI-03                                                                                                                            \\ \hline
\textbf{Nombre}        & Desactivar/Activar música                                                                                                           \\ \hline
\textbf{Requerimiento} & Se podrá desactivar la música ambiente del juego                                                                                    \\ \hline
\textbf{Descripción}   & \begin{tabular}[c]{@{}l@{}}Clicando en el icono de Música en el menú de opciones se\\   activará/desactivará la música\end{tabular} \\ \hline
\textbf{Prioridad}     & Baja                                                                                                                                \\ \hline
\end{tabular}
\end{table}

\begin{table}[H]
\centering
\label{my-label}
\begin{tabular}{|l|l|}
\hline
\textbf{Identificador} & RF-UI-04                                                                                                                                           \\ \hline
\textbf{Nombre}        & Desactivar/Activar sonidos                                                                                                                         \\ \hline
\textbf{Requerimiento} & Se podrán desactivar los efectos de sonido del juego                                                                                               \\ \hline
\textbf{Descripción}   & \begin{tabular}[c]{@{}l@{}}Clicando en el icono de Sonidos en el menú de opciones se\\   activarán/desactivarán los efectos de sonido\end{tabular} \\ \hline
\textbf{Prioridad}     & Baja                                                                                                                                               \\ \hline
\end{tabular}
\end{table}

\begin{table}[H]
\centering
\label{my-label}
\begin{tabular}{|l|l|}
\hline
\textbf{Identificador} & RF-UI-05                                                                                                                                          \\ \hline
\textbf{Nombre}        & Eliminar logros                                                                                                                                   \\ \hline
\textbf{Requerimiento} & Se podrán eliminar los logros conseguidos                                                                                                         \\ \hline
\textbf{Descripción}   & \begin{tabular}[c]{@{}l@{}}Clicando en el icono de Eliminar logros del menú de opciones se\\   eliminarán todos los logros obtenidos\end{tabular} \\ \hline
\textbf{Prioridad}     & Baja                                                                                                                                              \\ \hline
\end{tabular}
\end{table}

\begin{table}[H]
\centering
\label{my-label}
\begin{tabular}{|l|l|}
\hline
\textbf{Identificador} & RF-UI-06                                                                                                                                                              \\ \hline
\textbf{Nombre}        & Ver otros juegos                                                                                                                                                      \\ \hline
\textbf{Requerimiento} & Se podrá acceder a la descarga de otros juegos del autor                                                                                                              \\ \hline
\textbf{Descripción}   & \begin{tabular}[c]{@{}l@{}}Clicando en los iconos de juegos del apartado "Mas juegos" se\\   accederá a Google Play donde se podrá descargar dicho juego\end{tabular} \\ \hline
\textbf{Prioridad}     & Baja                                                                                                                                                                  \\ \hline
\end{tabular}
\end{table}

\begin{table}[H]
\centering
\label{my-label}
\begin{tabular}{|l|l|}
\hline
\textbf{Identificador} & RF-UI-07                                                                                                                                                                           \\ \hline
\textbf{Nombre}        & Navegar entre logros                                                                                                                                                               \\ \hline
\textbf{Requerimiento} & Se podrán seleccionar logros y ver su descripción                                                                                                                                  \\ \hline
\textbf{Descripción}   & \begin{tabular}[c]{@{}l@{}}En el menú de logros se podrá navegar entre los logros usando las flechas\\   de navegación y ver la descripción y la imagen de los mismos\end{tabular} \\ \hline
\textbf{Prioridad}     & Media                                                                                                                                                                              \\ \hline
\end{tabular}
\end{table}

\begin{table}[H]
\centering
\label{my-label}
\begin{tabular}{|l|l|}
\hline
\textbf{Identificador} & RF-UI-08                                                                                                                                                                    \\ \hline
\textbf{Nombre}        & Ver logros desbloquados                                                                                                                                                     \\ \hline
\textbf{Requerimiento} & Se podrá ver la cantidad de logros desbloqueados hasta el momento                                                                                                           \\ \hline
\textbf{Descripción}   & \begin{tabular}[c]{@{}l@{}}En el menú de logros aparecerá un texto representando el número de logros\\   desbloqueados hasta el momento y el total a conseguir\end{tabular} \\ \hline
\textbf{Prioridad}     & Baja                                                                                                                                                                        \\ \hline
\end{tabular}
\end{table}

\begin{table}[H]
\centering
\label{my-label}
\begin{tabular}{|l|l|}
\hline
\textbf{Identificador} & RF-UI-09                                                                                                                  \\ \hline
\textbf{Nombre}        & Ver descripción de logro                                                                                                  \\ \hline
\textbf{Requerimiento} & Se podrá ver una descripción de los logros                                                                                \\ \hline
\textbf{Descripción}   & \begin{tabular}[c]{@{}l@{}}En el menú de logros se podrá ver un texto descriptivo del logro\\   seleccionado\end{tabular} \\ \hline
\textbf{Prioridad}     & Media                                                                                                                     \\ \hline
\end{tabular}
\end{table}

\begin{table}[H]
\centering
\label{my-label}
\begin{tabular}{|l|l|}
\hline
\textbf{Identificador} & RF-UI-10                                                                                                                                                                                                               \\ \hline
\textbf{Nombre}        & Navegar entre logros                                                                                                                                                                                                   \\ \hline
\textbf{Requerimiento} & Se podrá seleccionar el logro del que se desea ver información                                                                                                                                                         \\ \hline
\textbf{Descripción}   & \begin{tabular}[c]{@{}l@{}}En el menú de logros se podrá navegar entre logros tocando y arrastrando\\   en la lista de logros. El logro que se sitúe en el centro de la pantalla será\\   el logro activo\end{tabular} \\ \hline
\textbf{Prioridad}     & Media                                                                                                                                                                                                                  \\ \hline
\end{tabular}
\end{table}

\begin{table}[H]
\centering
\label{my-label}
\begin{tabular}{|l|l|}
\hline
\textbf{Identificador} & RF-UI-11                                                                                                                                                                                                                                         \\ \hline
\textbf{Nombre}        & Abrir selector de personaje                                                                                                                                                                                                                      \\ \hline
\textbf{Requerimiento} & Se podrá abrir un menú donde seleccionar el personaje a controlar                                                                                                                                                                                \\ \hline
\textbf{Descripción}   & \begin{tabular}[c]{@{}l@{}}Clicando en el icono del selector de personaje se abrirá una ventana con\\   los iconos de cada personaje. El icono del personaje actualmente controlado\\   se mostrará en gris y no será seleccionable\end{tabular} \\ \hline
\textbf{Prioridad}     & Imprescindible                                                                                                                                                                                                                                   \\ \hline
\end{tabular}
\end{table}

\begin{table}[H]
\centering
\label{my-label}
\begin{tabular}{|l|l|}
\hline
\textbf{Identificador} & RF-UI-12                                                                                                                                                                                                                 \\ \hline
\textbf{Nombre}        & Seleccionar personaje                                                                                                                                                                                                    \\ \hline
\textbf{Requerimiento} & Al clicar en un icono de personaje se cambiará el personaje seleccionado                                                                                                                                                 \\ \hline
\textbf{Descripción}   & \begin{tabular}[c]{@{}l@{}}Al clicar en uno de los iconos de personaje se cambia el personaje\\   controlado. En el icono del selector de personaje aparecerá el icono del\\   nuevo personaje seleccionado\end{tabular} \\ \hline
\textbf{Prioridad}     & Imprescindible                                                                                                                                                                                                           \\ \hline
\end{tabular}
\end{table}

\begin{table}[H]
\centering
\label{my-label}
\begin{tabular}{|l|l|}
\hline
\textbf{Identificador} & RF-UI-13                                                              \\ \hline
\textbf{Nombre}        & Cerrar selector de personaje                                          \\ \hline
\textbf{Requerimiento} & Se podrá cerrar el menú donde seleccionar el personaje a controlar    \\ \hline
\textbf{Descripción}   & Clicando en el selector de personaje, si este está abierto se cerrará \\ \hline
\textbf{Prioridad}     & Imprescindible                                                        \\ \hline
\end{tabular}
\end{table}

\begin{table}[H]
\centering
\label{my-label}
\begin{tabular}{|l|l|}
\hline
\textbf{Identificador} & RF-UI-14                                                                                                                                                                                                                        \\ \hline
\textbf{Nombre}        & Notificar objetivo completado                                                                                                                                                                                                   \\ \hline
\textbf{Requerimiento} & Cuando se cumpla el objetivo actual se mostrará una notificación                                                                                                                                                                \\ \hline
\textbf{Descripción}   & \begin{tabular}[c]{@{}l@{}}Al completar un objetivo el icono del indicador de objetivos girará. El\\   texto del nuevo objetivo se añadirá a la lista de objetivos. El objetivo\\   completado se mostrará tachado\end{tabular} \\ \hline
\textbf{Prioridad}     & Media                                                                                                                                                                                                                           \\ \hline
\end{tabular}
\end{table}

\begin{table}[H]
\centering
\label{my-label}
\begin{tabular}{|l|l|}
\hline
\textbf{Identificador} & RF-UI-15                                                                                                                                                                                    \\ \hline
\textbf{Nombre}        & Abrir indicador de objetivos                                                                                                                                                                \\ \hline
\textbf{Requerimiento} & Se podrá abrir la lista de objetivos                                                                                                                                                        \\ \hline
\textbf{Descripción}   & \begin{tabular}[c]{@{}l@{}}Al clicar en el icono del indicador de objetivos se desplegará una lista\\   con los objetivos completados tachados y el objetivo actual sin tachar\end{tabular} \\ \hline
\textbf{Prioridad}     & Media                                                                                                                                                                                       \\ \hline
\end{tabular}
\end{table}

\begin{table}[H]
\centering
\label{my-label}
\begin{tabular}{|l|l|}
\hline
\textbf{Identificador} & RF-UI-16                                                                                                                                \\ \hline
\textbf{Nombre}        & Cerrar indicador de objetivos                                                                                                           \\ \hline
\textbf{Requerimiento} & Se podrá cerrar la lista de objetivos                                                                                                   \\ \hline
\textbf{Descripción}   & \begin{tabular}[c]{@{}l@{}}Al clicar en el icono del indicador de objetivos se cerrará la lista si\\   esta estaba abierta\end{tabular} \\ \hline
\textbf{Prioridad}     & Media                                                                                                                                   \\ \hline
\end{tabular}
\end{table}

\begin{table}[H]
\centering
\label{my-label}
\begin{tabular}{|l|l|}
\hline
\textbf{Identificador} & RF-UI-17                                                                                                                                                                                                                                       \\ \hline
\textbf{Nombre}        & Abrir menú conversacional                                                                                                                                                                                                                      \\ \hline
\textbf{Requerimiento} & Se abrirá el menú conversacional al interactuar con un personaje                                                                                                                                                                               \\ \hline
\textbf{Descripción}   & \begin{tabular}[c]{@{}l@{}}Al abrir el menú conversacional se hará zoom sobre los personajes\\   involucrados. Se mostrará un texto en la pantalla correspondiente a lo que\\   dice el NPC y botones con las posibles respuestas\end{tabular} \\ \hline
\textbf{Prioridad}     & Imprescindible                                                                                                                                                                                                                                 \\ \hline
\end{tabular}
\end{table}

\begin{table}[H]
\centering
\label{my-label}
\begin{tabular}{|l|l|}
\hline
\textbf{Identificador} & RF-UI-18                                                                                                                         \\ \hline
\textbf{Nombre}        & Contestar NPC                                                                                                                    \\ \hline
\textbf{Requerimiento} & Se podrá contestar a los NPC usando el menú conversacional                                                                       \\ \hline
\textbf{Descripción}   & \begin{tabular}[c]{@{}l@{}}Clicando en uno de los botones se contestará al NPC y se actualizará la\\   conversación\end{tabular} \\ \hline
\textbf{Prioridad}     & Imprescindible                                                                                                                   \\ \hline
\end{tabular}
\end{table}

\begin{table}[H]
\centering
\label{my-label}
\begin{tabular}{|l|l|}
\hline
\textbf{Identificador} & RF-UI-19                                                                                                                                                                                                                                                                                             \\ \hline
\textbf{Nombre}        & Terminar conversación                                                                                                                                                                                                                                                                                \\ \hline
\textbf{Requerimiento} & \begin{tabular}[c]{@{}l@{}}Al alcanzar cierto punto de la conversación se cerrará el menu\\   conversacional\end{tabular}                                                                                                                                                                            \\ \hline
\textbf{Descripción}   & \begin{tabular}[c]{@{}l@{}}Cuando la última decisión tomada tiene asociada tiene la marca\\   correspondiente de fin de conversación el menú conversacional se cerrará. El\\   punto en el que se encuentra la conversación se guarda para retomarla desde\\   ese punto posteriormente\end{tabular} \\ \hline
\textbf{Prioridad}     & Imprescindible                                                                                                                                                                                                                                                                                       \\ \hline
\end{tabular}
\end{table}

\subsection{Requisitos de sistema}

\begin{table}[H]
\centering
\label{my-label}
\begin{tabular}{|l|l|}
\hline
\textbf{Identificador} & RF-SYS-01                                                                                                                                                    \\ \hline
\textbf{Nombre}        & Iniciar menú principal                                                                                                                                       \\ \hline
\textbf{Requerimiento} & \begin{tabular}[c]{@{}l@{}}Al iniciar el juego el menú principal comienza en la vista\\   correspondiente\end{tabular}                                       \\ \hline
\textbf{Descripción}   & \begin{tabular}[c]{@{}l@{}}Al iniciar el juego el menú principal mostrará la vista con el título\\   principal y el botón para iniciar el juego\end{tabular} \\ \hline
\textbf{Prioridad}     & Imprescindible                                                                                                                                               \\ \hline
\end{tabular}
\end{table}

\begin{table}[H]
\centering
\label{my-label}
\begin{tabular}{|l|l|}
\hline
\textbf{Identificador} & RF-SYS-02                                                                                                                                                                                                                                         \\ \hline
\textbf{Nombre}        & Seleccionar personaje                                                                                                                                                                                                                             \\ \hline
\textbf{Requerimiento} & \begin{tabular}[c]{@{}l@{}}Al clicar en un icono del selector de personaje se cambiará el personaje\\   controlado\end{tabular}                                                                                                                   \\ \hline
\textbf{Descripción}   & \begin{tabular}[c]{@{}l@{}}Cuando se clique en un icono del selector de personaje se moverá la\\   cámara hasta enfocar al nuevo personaje seleccionado. Cuando se clique en la\\   pantalla se moverá el nuevo personaje controlado\end{tabular} \\ \hline
\textbf{Prioridad}     & Imprescindible                                                                                                                                                                                                                                    \\ \hline
\end{tabular}
\end{table}

\begin{table}[H]
\centering
\label{my-label}
\begin{tabular}{|l|l|}
\hline
\textbf{Identificador} & RF-SYS-03                                                                                                                                                                                                                                                   \\ \hline
\textbf{Nombre}        & Cargar conversación                                                                                                                                                                                                                                         \\ \hline
\textbf{Requerimiento} & Se cargarán las conversaciones desde archivos de texto                                                                                                                                                                                                      \\ \hline
\textbf{Descripción}   & \begin{tabular}[c]{@{}l@{}}Al abrir el menú conversacional se cargará el fichero de texto\\   correspondiente a la conversación del personaje. Si ya se ha conversado con\\   ese personaje se cargará la conversación desde el punto guardado\end{tabular} \\ \hline
\textbf{Prioridad}     & Imprescindible                                                                                                                                                                                                                                              \\ \hline
\end{tabular}
\end{table}

\begin{table}[H]
\centering
\label{my-label}
\begin{tabular}{|l|l|}
\hline
\textbf{Identificador} & RF-SYS-04                                                                                                                                                                        \\ \hline
\textbf{Nombre}        & Contestar NPC                                                                                                                                                                    \\ \hline
\textbf{Requerimiento} & Al contestar a un NPC se actualizarán las respuestas y el texto del NPC                                                                                                          \\ \hline
\textbf{Descripción}   & \begin{tabular}[c]{@{}l@{}}Al contestar a un NPC se leerá del JSON el nuevo texto y las nuevas\\   respuestas y se actualizará la interfaz para mostrar estos datos\end{tabular} \\ \hline
\textbf{Prioridad}     & Imprescindible                                                                                                                                                                   \\ \hline
\end{tabular}
\end{table}

\begin{table}[H]
\centering
\label{my-label}
\begin{tabular}{|l|l|}
\hline
\textbf{Identificador} & RF-SYS-05                                                                                                                                                                                                                                                                                                                 \\ \hline
\textbf{Nombre}        & Codificar conversaciones                                                                                                                                                                                                                                                                                                  \\ \hline
\textbf{Requerimiento} & \begin{tabular}[c]{@{}l@{}}Las conversaciones estarán codificadas en formato JSON y guardadas en\\   ficheros de texto\end{tabular}                                                                                                                                                                                       \\ \hline
\textbf{Descripción}   & \begin{tabular}[c]{@{}l@{}}Cada NPC tendrá un fichero de texto asociado con la conversación que\\   ofrece en formato JSON. Los elementos del JSON serán nodos con texto y nodos\\   hijos con las respuestas asociadas a ese texto. Las respuestas podrán tener\\   "markups" asociadas que indican eventos\end{tabular} \\ \hline
\textbf{Prioridad}     & Imprescindible                                                                                                                                                                                                                                                                                                            \\ \hline
\end{tabular}
\end{table}

\begin{table}[H]
\centering
\label{my-label}
\begin{tabular}{|l|l|}
\hline
\textbf{Identificador} & RF-SYS-06                                                                                                                                                 \\ \hline
\textbf{Nombre}        & Completar objetivo                                                                                                                                        \\ \hline
\textbf{Requerimiento} & \begin{tabular}[c]{@{}l@{}}Se completarán objetivos al tomar decisiones con la markup\\   correspondiente\end{tabular}                                    \\ \hline
\textbf{Descripción}   & \begin{tabular}[c]{@{}l@{}}Cuando se elige una opción de conversación con la "markup" de\\   objetivo asociada, se completará dicho objetivo\end{tabular} \\ \hline
\textbf{Prioridad}     & Imprescindible                                                                                                                                            \\ \hline
\end{tabular}
\end{table}

\begin{table}[H]
\centering
\label{my-label}
\begin{tabular}{|l|l|}
\hline
\textbf{Identificador} & RF-SYS-07                                                                                                                                                                                                               \\ \hline
\textbf{Nombre}        & Cambiar de escenario                                                                                                                                                                                                    \\ \hline
\textbf{Requerimiento} & Cuando el jugador cambia de escenario se modifica la escena                                                                                                                                                             \\ \hline
\textbf{Descripción}   & \begin{tabular}[c]{@{}l@{}}Cuando se produzca un cambio de escenario se desactivará de la escena el\\   escenario abandonado y el jugador aparecerá en el punto de "spawn"\\   asociado al nuevo escenario\end{tabular} \\ \hline
\textbf{Prioridad}     & Imprescindible                                                                                                                                                                                                          \\ \hline
\end{tabular}
\end{table}

\begin{table}[H]
\centering
\label{my-label}
\begin{tabular}{|l|l|}
\hline
\textbf{Identificador} & RF-SYS-08                                                                                                                                                               \\ \hline
\textbf{Nombre}        & Guardar juego                                                                                                                                                           \\ \hline
\textbf{Requerimiento} & El estado del juego se guardará al salir                                                                                                                                \\ \hline
\textbf{Descripción}   & \begin{tabular}[c]{@{}l@{}}Cuando el usuario salga de la aplicación o al menú principal, el estado\\   del juego se guardará en la memoria del dispositivo\end{tabular} \\ \hline
\textbf{Prioridad}     & Baja                                                                                                                                                                    \\ \hline
\end{tabular}
\end{table}

\begin{table}[H]
\centering
\label{my-label}
\begin{tabular}{|l|l|}
\hline
\textbf{Identificador} & RF-SYS-09                                                                                                                                                                                                                    \\ \hline
\textbf{Nombre}        & Cargar juego                                                                                                                                                                                                                 \\ \hline
\textbf{Requerimiento} & Al iniciar el juego se cargarán las partidas guardadas                                                                                                                                                                       \\ \hline
\textbf{Descripción}   & \begin{tabular}[c]{@{}l@{}}Cuando se inice el juego se cargarán las partidas guardadas en caso de\\   existan. De ser así, al iniciar la partida el mundo se encontrará en el\\   estado de la partida guardada\end{tabular} \\ \hline
\textbf{Prioridad}     & Baja                                                                                                                                                                                                                         \\ \hline
\end{tabular}
\end{table}

\begin{table}[H]
\centering
\label{my-label}
\begin{tabular}{|l|l|}
\hline
\textbf{Identificador} & RF-SYS-10                                                                                                                                                                                                                                                                            \\ \hline
\textbf{Nombre}        & Actualizar conversaciones                                                                                                                                                                                                                                                            \\ \hline
\textbf{Requerimiento} & Al completar un objetivo se actualizan las conversaciones                                                                                                                                                                                                                            \\ \hline
\textbf{Descripción}   & \begin{tabular}[c]{@{}l@{}}Al completar un objetivo en una conversación, todas las demás\\   conversaciones se actualizarán a un punto correspondiente de la misma para\\   reflejar el objetivo completado. Las nuevas conversaciones empezarán desde\\   dicho punto.\end{tabular} \\ \hline
\textbf{Prioridad}     & Imprescindible                                                                                                                                                                                                                                                                       \\ \hline
\end{tabular}
\end{table}

\section{Requisitos no funcionales}

\begin{itemize}

\item Rendimiento: el sistema debe de proporcionar una experiencia de juego óptima. Para ello el flujo de juego debe de ser fluido, sin parones o interrupciones.
\item Usabilidad: el sistema debe de ser sencillo de utilizar. De modo que personas sin un contacto o entrenamiento previo sepan desenvolverse por el sistema
\item Costo: el coste del sistema deberá ser cero

\end{itemize}

\section{Arquitectura del sitema}

Para poder explicar la arquitectura del sistema construido es necesario en primer lugar exponer la arquitectura del motor de videojuegos utilizado. El motivo es que el sistema construido tendrá que cumplir con las normas y reglas del motor, ya que se usarán las herramientas que este provee. Es por ello que se puede afirmar que la arquitectura del videojuego está condicionada y subordinada a la del motor utilizado.

Dado que para la realización de este proyecto se ha utilizado el motor de videojuegos Unity3D \footnote{https://unity3d.com/es/}, se explicará a continuación el diseño de dicho motor.

\subsection{Arquitectura de Unity3D}

Unity3D basa su diseño en el modelo entidad-componente \footnote{https://www.genbetadev.com/programacion-de-videojuegos/diseno-de-videojuegos-orientado-a-entidades-y-componentes}. En este diseño, las entidades del mundo del juego obtienen su funcionalidad mediante la agregación de diferentes componentes. 

En el paradigma de la programación orientada a objetos \footnote{https://desarrolloweb.com/articulos/499.php} el comportamiento de una entidad se define en la clase que la representa. Sin embargo, para reducir la duplicación de código se emplea el mecanismo de la herencia, por el cual una clase puede heredar de otra y de esta forma obtener su comportamiento y ampliarlo. 
El problema que reside en basar el diseño en jerarquías de herencia es que a medida que crece la complejidad del sistema crece el "árbol" que forman las clases. Llegados a este punto es probable que se de el caso en el que se precisa cambiar la funcionalidad de una de las clases, y que al hacerlo se modifique de forma involuntaria el comportamiento de otras clases que heredan de la modificada. Llevando el caso a un punto más extremo, se podría dar la situación en la que el comportamiento de una clase sea incompatible con el comportamiento heredado de una clase padre.

Para solucionar este problema el diseño entidad-componente propone crear entidades que actúen como meros contenedores. Dichos contenedores agregarán componentes que les darán la funcionalidad que los definirá. Por ejemplo tanto un jugador como los objetos del escenario tendrían el componente Mesh (malla gráfica) pero solo el jugador tendría el componente Movement (movimiento). Como se puede observar este diseño es altamente flexible, ya que si se toma la precaución de hacer los componentes lo suficientemente genéricos, se pueden aplicar a varias entidades, de forma que hay una gran reutilización de código. 

Otra ventaja es que se produce un muy bajo acoplamiento ya que las funcionalidades están encapsuladas en los componentes y estos son independientes entre sí.

La forma en la que Unity3D aplica este diseño entidad-componente es mediante el uso de \textquote{Gameobjects} (entidades) y components.
Los \textquote{Gameobjects} son la entidad fundamental en Unity3D: todos los elementos que existan en la escena son un \textquote{Gameobject}. Además todas las entidades en Unity3D (en adelante Gameobjects) incorporan como mínimo un componente. Dicho componente se llama \textquote{Transform} y se encarga de la gestión de la posición, la rotación y la escala del \textquote{Gameobject} que lo posee.

En la siguiente imagen (ver fig. \ref{gameobjectComponents}) se puede observar los componentes que posee el \textquote{Gameobject} Leonardo. 

\begin{figure}
\begin{center}
\includegraphics[scale=0.9]{imagenes/GameobjectsAndComponents.png}
\caption{Vista del editor de Unity3D donde se pueden ver los componentes de un Gameobject.  Fuente: elaboración propia}
\label{gameobjectComponents}
\end{center}
\end{figure}

Como se ha mencionado anteriormente el componente \textquote{Transform} es obligatorio en todos \textquote{Gameobjects}. También se pueden ver otros dos componentes, \textquote{Movement Manager} y \textquote{Player}, que han sido creados por el desarrollador.
Dichos componentes dotan a la entidad, que en este caso es el personaje controlable por el jugador, de las características que lo hacen único. Entre otras cosas le dan la habilidad de reaccionar ante los inputs del usuario y moverse por el escenario. También se pueden observar una serie de variables con parámetros. Dado que cada instancia de los componentes es única para la entidad que lo agrega, se pueden personalizar para que incluso entre entidades con los mismos componentes se comporten de manera diferente.

En la siguiente imagen (ver fig. \ref{diagramaClasesUnity}) se puede observar un diagrama de clases que muestra la estructura que compone el sistema de componentes y los Gameobjects.

\begin{figure}
\begin{center}
\includegraphics[scale=0.8]{imagenes/diagramaClasesUnity.png}
\caption{Diagrama de clases de Unity3D.  Fuente: elaboración propia}
\label{diagramaClasesUnity}
\end{center}
\end{figure}

Los componentes que provee Unity3D derivan de la clase Component. En el caso de la imagen solo aparece el componente \textquote{Transform}, pero hay muchos más como \textquote{Renderer}, \textquote{Rigidbody}, \textquote{Collider}, etc. En la clase especializada cada componente contendrá la lógica necesaria para desempeñar su función, y el la clase Component se referenciará al \textquote{GameObject} que lo contiene. Cabe notar que un componente no puede existir de forma independiente. Debe estar siempre asociado a una entidad.

La clase MonoBehaviour es también muy importante ya que es la clase base para los scripts creados por los desarrollares. Cualquier fragmento de código que se desee añadir como componente a una entidad deberá heredar de dicha clase. 

Por supuesto se pueden escribir scripts que no hereden de la clase MonoBehaviour, pero en ese caso no podrán ser asignados como componentes a una entidad. Sin embargo no dejan de ser útiles ya que pueden ser usados como POCO \footnote{https://es.wikipedia.org/wiki/Plain\_Old\_CLR\_Object}, definir interfaces, contener lógica o cualquier otra cosa que el desarrollador desee. 

En última instancia los scripts que no heredan de MonoBehaviour deben de ser utilizados desde un script que sí lo haga, ya que este es el único punto de entrada que tienen hacia el sistema de Unity3D.

El último aspecto de la arquitectura que se comentará es el bucle de juego: una parte fundamenta de cualquier videojuego.
El bucle de juego es la parte software más importante en cualquier videojuego. En él se ejecutan las tareas que hacen que el juego 'esté vivo'.
En el siguiente fragmento de código (ver fig. \ref{bucleDeJuego}) se presenta un ejemplo de bucle de juego básico:

\begin{lstlisting}[caption={Código de bucle de juego básico},label=bucleDeJuego]
while(true)
{
	ProcesarInput();
	ActualizarJuego();
	Renderizar();
}
\end{lstlisting}

En este fragmento de código se pueden observar las tres tareas fundamentales de las que se compone cualquier videojuego:

\begin{itemize}
	\item Procesar input: consiste en capturar la interacción del usuario con el sistema mediante el hardware. En esta etapa se puede aplicar algún tipo de filtrado sobre dicho input.
	\item Actualizar juego: se actualiza el estado del juego. Esto consiste en actualizar IA, realizar las acciones del personaje, actualizar el mundo, efectos de objetos, etc.
	\item Renderizar: los elementos del juego se procesan para ser dibujados en la pantalla.
\end{itemize}

El ejemplo de bucle mostrado anteriormente es tremendamente básico pero sirve para explicar las bases de un bucle de juego. Probablemente ningún videojuego moderno lo utilice ya que tiene múltiples inconvenientes como que la frecuencia de actualización del juego dependerá de la carga de trabajo y del rendimiento del sistema que la procese. Actualmente existen variaciones del bucle que solucionan este problema, pero son temas que quedan fuera del alcance de este proyecto.

En la siguiente imagen (ver fig. \ref{bucleUnity}) se puede observar el bucle de juego utilizado por el motor Unity3D, y evidentemente por el juego creado para este proyecto:

\begin{figure}
\begin{center}
\includegraphics[scale=0.55]{imagenes/bucleUnity.png}
\caption{Bucle de juego del motor Unity3D. Fuente: https://docs.unity3d.com/Manual/ExecutionOrder.html}
\label{bucleUnity}
\end{center}
\end{figure}

En la imagen se pueden observar las 3 fases clave del bucle de juego: la captura de acciones del jugador, que se produce en la parte denominada como \textquote{Input Events}; la actualización del mundo de juego que se lleva a cabo en \textquote{Physics} y \textquote{Game logic}; finalmente el dibujado del mundo, que se produce en \textquote{Scene rendering}, \textquote{Gizmo rendering} y \textquote{GUI rendering}.


\subsection{Arquitectura del juego}

A continuación se describirá la arquitectura utilizada para desarrollar los componentes más importantes del juego. Estos son: el sistema de conversaciones y el sistema de escenarios. %y el sistema de cambio de personajes , el sistema de movimiento,

\subsubsection{Sistema de conversaciones}

El sistema de conversaciones es el más importante del juego ya que los diálogos con los personajes del juego son el hilo conductor de este.

Los diálogos en lugar de estar insertados directamente en el código se almacenan en ficheros de texto plano independientes. Esto permite que se puedan modificar los diálogos sin tener que recompilar el código del juego. Además permite que cualquier persona sin conocimientos de programación escriba diálogos.

Cada diálogo es un archivo que sigue el formato JSON \footnote{https://geekytheory.com/json-i-que-es-y-para-que-sirve-json}. Las conversaciones tienen una estructura arbórea donde los nodos representan la parte del diálogo correspondiente al NPC y las ramas salientes de dicho nodo representan las posibles contestaciones. La representación de una conversación en el formato JSON sigue la estructura que se puede ver en la siguiente imagen (ver fig. \ref{jsonFormat}).

\begin{figure}
\begin{center}
\includegraphics[scale=0.57]{imagenes/jsonFormat.png}
\caption{Representación del formato utilizado para los ficheros de diálogo. Fuente: elaboración propia}
\label{jsonFormat}
\end{center}
\end{figure}

Como se puede observar, los datos contenidos en el JSON son un array de objetos Nodo y un string que indica el nodo inicial de la conversación. 

Los objetos Nodos están indentificados por una cadena de texto que se compone de las tres primeras palabras del diálogo.Cada Nodo contiene el texto del NPC y las posibles contestaciones que el jugador le puede dar.

Las contestaciones se componen de el identificador del nodo al que conduce dicha contestación y el texto de la contestación en sí. Además puede contener ,o no, diferentes \textquote{flags} que le dan a la elección una funcionalidad extra. Entre las \textquote{flags} disponibles se encuentran la de acabar la conversación, desbloquear un objetivo o desbloquear un logro.

Para la creación de los ficheros de diálogos se puede escribir el JSON manualmente rellenando con los datos deseados o se puede utilizar la herramienta inklewritter \footnote{http://www.inklestudios.com/inklewriter/}. Dicha herramienta presenta una forma más sencilla de escribir conversaciones de múltiple respuesta (ver fig. \ref{inklewritter}). Además cuenta con la posibilidad de exportar el trabajo final a un fichero de texto en formato JSON. El parser JSON del videojuego está construido específicamente para poder procesar los diálogos construidos con inklewritter.

\begin{figure}
\begin{center}
\includegraphics[scale=0.7]{imagenes/inklewritter.png}
\caption{Herramienta inklewritter. Fuente: elaboración propia}
\label{inklewritter}
\end{center}
\end{figure}

Para explicar la arquitectura del sistema de conversaciones se emplearán sendos diagramas de secuencia: uno para explicar el proceso de iniciar una conversación y otro para explicar el proceso de selección de contestaciones.

En el siguiente diagrama (ver fig. \ref{nuevaConversFlujo}) se ve el flujo de código necesario para iniciar una conversación. 

\begin{figure}
\begin{center}
\includegraphics[scale=0.55]{imagenes/nuevaConversFlujo.png}
\caption{Diagrama de flujo nueva conversación.  Fuente: elaboración propia}
\label{nuevaConversFlujo}
\end{center}
\end{figure}

No se ha indicado en el diagrama pero para el tratamiento JSON del fichero y para todas las operaciones JSON se utiliza la librería SimpleJSON \footnote{http://wiki.unity3d.com/index.php/SimpleJSON}.


El punto de entrada es el click del jugador en un NPC. Previamente el NPC se habrá registrado en el ConversationManager, de forma que el NPC guarda en un delegado el método NewConversation(string) del ConversationManager.

Cuando se le pide al ConversationManager que inicie una nueva conversación, el NPC le pasa por parámetro la ruta del fichero en la que está almacenado su diálogo con el personaje que está controlando el jugador en ese momento. El ConversationManager informará al JSONParser de que se debe iniciar una nueva conversación, y este será el encargado de leer y parsear el fichero o de abrir una de las conversaciones que tiene almacenadas.

Una vez informado el JSONParser, el ConversationManager le pedirá el texto del nodo actual de la conversación y el texto de las posibles respuestas. Una vez tenga esa información se la pasará al UIManager que se encarga de la interacción con los elementos gráficos de la interfaz.

En este arquitectura se ha seguido un diseño inspirado en el patrón Model-view-Controller \footnote{https://es.wikipedia.org/wiki/Modelo-vista-controlador}. Para ello todas las referencias a los elementos de la interfaz y los métodos de interacción con los mismos se han encapsulado en la clase llama UIManager. El acceso a los datos de las conversaciones, la gestión del estado de las mismas y la lectura de los ficheros de diálogos se guardan en la clase JSONParser, que además utiliza la librería SimpleJSON. En cuanto al punto de encuentro entre los datos y la interfaz, se emplea la clase ConversationManager para sincronizar las operaciones entre las otras dos clases y para proveer un punto de entrada al sistema de conversaciones.

Una vez abierta la conversación es necesario actualizar el juego cada vez que el jugador selecciona una respuesta. El funcionamiento de esta mecánica es muy similar y utiliza los mismos componentes que el sistema anterior.
En el siguiente diagrama de flujo (ver fig. \ref{respuestaFlujo}) se explica el proceso.

\begin{figure}
\begin{center}
\includegraphics[scale=0.55]{imagenes/respuestaFlujo.png}
\caption{Diagrama de flujo de selección de respuesta.  Fuente: elaboración propia}
\label{respuestaFlujo}
\end{center}
\end{figure}

\subsubsection{Sistema de escenarios}

A lo largo del juego se puede navegar por diferentes escenarios. Por temas de eficiencia y debido a que cada escenario tiene características únicas se han agrupado según la temática. En lugar de tener un gran escenario con montones de mallas correspondientes a los edificios y los personajes, solo están activos los Gameobjects correspondientes al escenario actual, los demás están desactivados (que no eliminados). En los escenarios hay puertas que al ser clicadas transportan al personaje activo hacia el escenario al que conduce la puerta. 

En el inspector de escena de Unity3D dichos escenarios lucen como se en la siguiente imagen (ver fig. \ref{unityScenarios}).

\begin{figure}
\begin{center}
\includegraphics[scale=1]{imagenes/unityScenarios.png}
\caption{Vista de los escenarios en el inspector de Unity3D.  Fuente: elaboración propia}
\label{unityScenarios}
\end{center}
\end{figure}

Como se puede observar algunos Gameobjects de la escena contienen scripts y componentes (otros Gameobjects tambien pero se han omitido para mayor simplicidad). En el siguiente diagrama de clases (ver fig. \ref{sceneClassDiagram}) se muestra la estructura de estos scripts y posteriormente se explicará su funcionamiento y utilidad.

\begin{figure}
\begin{center}
\includegraphics[scale=0.75]{imagenes/sceneClassDiagram.PNG}
\caption{Diagrama de clases del sistema de escenarios.  Fuente: elaboración propia}
\label{sceneClassDiagram}
\end{center}
\end{figure}

El elemento central de todo el sistema de escenarios es la puerta. El nombre del Gameobject puerta sigue una notación que indica el escenario destino y origen. En el caso de la imagen anterior (ver fig. \ref{unityScenarios}) se puede observar que la puerta lleva del Palacio (el escenario activo) a Las calles (un escenario desactivado).

El método IPointerClickHandler.OnPointerClick(...), heredado de la interfaz IPointerClickHandler que provee Unity3D, es llamado cuando se hace click en el componente Collider asociado a la puerta. Posteriormente se comprobará que el click es válido (el jugador no está muy lejos por ejemplo) con el método heredado IsClickValid(). Finalmente se notificará al GameManager de que se quiere hacer un cambio de escena y se le suministrará el Gameobject que contiene la nueva escena y la posición de inicio para el jugador.

El sistema de escenarios se encarga también de gestionar la pista de audio que se reproduce a modo de música ambiental. En el objeto raíz de cada escenario se encuentra un script que hace uso del método provisto por Unity3D, OnEnable(). Dicho método es llamado cada vez que se activa un escenario. Desde el método OnEnable() se notifica al SoundManager, que es accesible de forma global ya que implementa el patrón Singleton, el nuevo escenario activado. El SoundManager dispone de un clip de sonido por cada escenario y cuando recibe el tipo del nuevo escenario cambia el clip de sonido reproducido por el componente AudioSource.

\subsection{Storytelling}

El storytelling es una parte fundamental del videojuego creado. Uno de los aspectos que hacen diferente a Da Vinci Startup es que en lugar de saturar al usuario con insufribles lecciones de teoría sobre emprendimiento, se aprende a la vez que se juega. Eso no significa que el conocimiento técnico no esté presente. La diferencia es que el conocimiento se adquiere a lo largo de la historia que los NPCs cuentan y que el jugador experimenta. De este modo conceptos como el "networking" no son solo palabras en un manual, si no que el jugador deberá utilizarlo a lo largo del juego para lograr los objetivos que le llevarán a completar el mismo.

\subsubsection{Guion literario}

Previamente a la implementación del videojuego se escribió un guion literario. Dicho guion explica la historia a desarrollar en el juego de una forma más narrativa, como si de un libro o historia se tratase.
Dicho guion se encuentra en el apartado \nameref{guionLiterario} incluido en el \nameref{GDD}. Dicha historia es una historia ficticia en la que un mentor guía a Leonardo durante la construcción de la startup.

\subsubsection{Guion técnico}

Además del guion literario se ha elaborado un guion técnico en el que se especifican las diferentes pantallas que componen el menú principal junto con los mockups que las definen. El diagrama de pantallas del menú principal se puede encontrar en el apartado \nameref{guionTecnico} del \nameref{GDD}.


Se han elaborado también los mockups que definen a los diferentes menús disponibles durante la partida. Se incluye además una explicación del funcionamiento de cada elemento que se puede consultar en el apartado \nameref{mockupsJuego} del Anexo I.


En cuanto a los personajes disponibles en el juego, tanto controlables como no controlables, y los escenarios se dispone de una descripción detallada en los apartados \nameref{personajes} y \nameref{descripcionEscenarios} respectivamente.
\chapter{Resultados}

\section{Section1}

blabla

\section{Section2}

bla bla


\chapter{Pruebas y validación}

Para el desarrollo no se ha optado por la escritura de tests automáticos ya que el tiempo requerido para este proceso habría sido inasumible. La utilización de tests automáticos es un proceso imprescindible en la ingeniería de software pero lamentablemente para este proyecto no ha sido posible.

Sin embargo sí que se han ejecutado diversas pruebas para comprobar que el sistema se comporta de forma adecuada y cumple con unos criterios de calidad.

\section{Pruebas exploratorias}

El la forma de testing más sencilla. Jugar el juego haciendo especial incapié en la búsqueda de bugs o errores. De esta forma se han encontrado multitud de comportamientos no deseados. Ha sido una de las formas de testing más utilizadas en este proyecto.

\section{Pruebas funcionales}

Haciendo uso de los requerimientos especificados en el apartado \nameref{requisitosUsuario} se ha comprobado que estaban correctamente implementados mediante la interacción directa con la aplicación final. De esta forma se validan dichos requisitos ya que se prueban tal cual lo haría un jugador.
Se han validado todos los requisitos implementados y se ha comprobado que funcionan de forma correcta.

\section{Pruebas de usabilidad}

Haciendo uso de un grupo de voluntarios que deseaban probar el juego se les dejó jugar durante un tiempo para luego realizarles una serie de preguntas. 
Se les pidió que valoraran la sencillez con la que se entendía las funcionalidades que provee la interfaz. Se les pidió también que valoraran la facilidad con la que el juego indicaba al jugador lo que debía hacer.
La última batería de pruebas fue pedir a los usuarios que realizaran tareas sencillas tales como cambiar de personaje o cambiar de escenario. El aspecto a evaluar era el tiempo que tardaban en completar la tarea y cuanto les había costado descubrir como realizarla.


En general el resultado de las pruebas fue satisfactorio. Las interfaces tienen el mínimo de elementos posibles y las mecánicas de juego tampoco son muy elevadas por lo tanto es sencillo desenvolverse con el sistema.

\section{Pruebas de compatibilidad}

El sistema se probó en diferentes dispositivos con tamaños de pantalla y resoluciones variadas. El objetivo era verificar que las proporciones del juego, la estructura de la interfaz y de los escenarios no se veian afectadas por la resolución del dispositivo. Los dispositivos probados fueron un Xiaomi redmi3 con el sistema operativo Android y una resolución de pantalla de 1920x1080 en 5.5 pulgadas y una tableta Nexus 7 con el sistema operativo Android y una resolución de 1920x1200 en 7 pulgadas


En ambos dispositivos los resultados fueron los esperados: la interfaz se adapta a la pantalla perfectamente. 



\chapter{Conclusiones}

\section{Mejoras y ampliaciones}

Como se ha indicado en el apartado anterior el producto creado no es un juego finalizado si no un Mínimo producto viable. Es por el ello que aún hay trabajo por delante hasta que se convierta en un producto comercializable

Las siguientes son algunas de las posibles mejoras al sistema:

\begin{itemize}
	\item Implementar el sistema de guardado y cargado de partida.
	\item Incrementar la calidad/complejidad/cantidad de los diálogos.
	\item Utilizar modelos 3D propios.
	\item Aplicar animaciones a los modelos 3D
\end{itemize}

\section{Modelo de negocio}




\begin{figure}[H]
\begin{center}
\includegraphics[scale=0.33]{imagenes/leanCanvasDaVinciStartup.png}
\caption{Lean canvas propuesto para este proyecto}
\label{leanCanvasDaVinci}
\end{center}
\end{figure}





%\nocite{*} %incluye TODOS los documentos de la base de datos bibliográfica sean o no citados en el texto
\bibliography{bibliografia/bibliografia}
\addcontentsline{toc}{chapter}{Bibliografía} %sustituir bibliografía con el nombre del fichero bibtex con la bibliografía
\bibliographystyle{apalike}
%
\appendix

% This file was converted to LaTeX by Writer2LaTeX ver. 1.4
% see http://writer2latex.sourceforge.net for more info
%\documentclass[a4paper]{article}
%\usepackage[ascii]{inputenc}
%\usepackage[T1]{fontenc}
%\usepackage[english,spanish]{babel}
%\usepackage{amsmath}
%\usepackage{amssymb,amsfonts,textcomp}
%\usepackage{color}
%\usepackage{array}
%\usepackage{hhline}
%\usepackage{hyperref}
%\usepackage{pdfpages}

%\usepackage{xcolor}

\hypersetup{pdftex, colorlinks=true, linkcolor=black, citecolor=blue, filecolor=blue, urlcolor=blue, pdftitle=DOCUMENTO DE DISE\~NO DE VIDEOJUEGO, pdfauthor=Alex Verd\'u, pdfsubject=Da Vinci startup, pdfkeywords=}
% Outline numbering
\setcounter{secnumdepth}{0}
% List styles
\newcounter{saveenum}
\newcommand\liststyleLFOxii{%
\renewcommand\theenumi{\arabic{enumi}}
\renewcommand\theenumii{\alph{enumii}}
\renewcommand\theenumiii{\roman{enumiii}}
\renewcommand\theenumiv{\arabic{enumiv}}
\renewcommand\labelenumi{\theenumi.}
\renewcommand\labelenumii{\theenumii.}
\renewcommand\labelenumiii{\theenumiii.}
\renewcommand\labelenumiv{\theenumiv.}
}
\newcommand\liststyleLFOxi{%
\renewcommand\labelitemi{{\textbullet}}
\renewcommand\labelitemii{o}
\renewcommand\labelitemiii{${\blacksquare}$}
\renewcommand\labelitemiv{{\textbullet}}
}
\newcommand\liststyleLFOvi{%
\renewcommand\theenumi{\arabic{enumi}}
\renewcommand\theenumii{\alph{enumii}}
\renewcommand\theenumiii{\roman{enumiii}}
\renewcommand\theenumiv{\arabic{enumiv}}
\renewcommand\labelenumi{\theenumi.}
\renewcommand\labelenumii{\theenumii.}
\renewcommand\labelenumiii{\theenumiii.}
\renewcommand\labelenumiv{\theenumiv.}
}
\newcommand\liststyleLFOvii{%
\renewcommand\theenumi{\arabic{enumi}}
\renewcommand\theenumii{\alph{enumii}}
\renewcommand\theenumiii{\roman{enumiii}}
\renewcommand\theenumiv{\arabic{enumiv}}
\renewcommand\labelenumi{\theenumi.}
\renewcommand\labelenumii{\theenumii.}
\renewcommand\labelenumiii{\theenumiii.}
\renewcommand\labelenumiv{\theenumiv.}
}
\newcommand\liststyleLFOxiii{%
\renewcommand\labelitemi{{\textbullet}}
\renewcommand\labelitemii{o}
\renewcommand\labelitemiii{${\blacksquare}$}
\renewcommand\labelitemiv{{\textbullet}}
}
\newcommand\liststyleLFOxiv{%
\renewcommand\labelitemi{{\textbullet}}
\renewcommand\labelitemii{o}
\renewcommand\labelitemiii{${\blacksquare}$}
\renewcommand\labelitemiv{{\textbullet}}
}
% Page layout (geometry)
\setlength\voffset{-1in}
\setlength\hoffset{-1in}
\setlength\topmargin{0.984in}
\setlength\oddsidemargin{1.1812in}
\setlength\textheight{9.724999in}
\setlength\textwidth{5.9055996in}
\setlength\footskip{0.0cm}
\setlength\headheight{0cm}
\setlength\headsep{0cm}
% Footnote rule
\setlength{\skip\footins}{1.1777999mm}
\renewcommand\footnoterule{\vspace*{-0.007in}\setlength\leftskip{0pt}\setlength\rightskip{0pt plus 1fil}\noindent\textcolor{black}{\rule{0.33\columnwidth}{0.007in}}\vspace*{1mm}}
% Pages styles
\makeatletter
\newcommand\ps@MP{
  \renewcommand\@oddhead{}
  \renewcommand\@evenhead{}
  \renewcommand\@oddfoot{}
  \renewcommand\@evenfoot{}
  \renewcommand\thepage{\arabic{page}}
}
\newcommand\ps@MPF{
  \renewcommand\@oddhead{}
  \renewcommand\@evenhead{}
  \renewcommand\@oddfoot{}
  \renewcommand\@evenfoot{}
  \renewcommand\thepage{\arabic{page}}
}
\makeatother
\pagestyle{MP}
\title{DOCUMENTO DE DISE\~NO DE VIDEOJUEGO}
\author{Alex Verd\'u}
\date{2017-06-07}
%\begin{document}
\chapter{Anexo I. Documento de diseño de videojuego}
\label{GDD}

\includepdf[pages={1},pagecommand={},fitpaper=true,trim=0 0 0 0, 
offset=0 0,turn=true,noautoscale=true]{anexos/GDD/portadaGDDImagen.pdf}


\clearpage\clearpage\setcounter{page}{1}\pagestyle{MP}
\thispagestyle{MPF}

\clearpage
\bigskip

\section[]{\selectlanguage{spanish} }
\section[Visi\'on general]{\selectlanguage{spanish} Visi\'on general}
\hypertarget{Toc484614208}{}\subsection[Introducci\'on]{\selectlanguage{spanish} Introducci\'on}
\hypertarget{Toc484614209}{}{\selectlanguage{spanish}
Este juego recrea una aventura ficticia del famoso inventor Leonardo da Vinci.\ }

{\selectlanguage{spanish}
A lo largo del juego,\ el inventor\ deber\'a desarrollar un producto, obtener financiaci\'on y venderlo siguiendo la
metodolog\'ia Lean Startup. Para ello contar\'a con los consejos de Andrea del\ Verrocchio que ser\'a su mentor en el
mundo del emprendimiento y las startups.}

{\selectlanguage{spanish}
El juego ser\'a una aventura gr\'afica al estilo del m\'itico juego Monkey island, aunque se desarrollar\'a en un
escenario 2.5D como en el t\'itulo Deadlight.}

\subsection[Tema]{\selectlanguage{spanish} Tema}
\hypertarget{Toc484614210}{}{\selectlanguage{spanish}
El juego se desarrolla en la Florencia\ renacentista, \'epoca en la que Leonardo trabaj\'o en el taller de Andrea de
Verrocchio. A lo largo de la partida se visitar\'an diversos escenarios como el taller de Andrea, las calles de
Florencia o los palacios de los burgueses.}

\subsection[Estilo visual]{\selectlanguage{spanish} Estilo visual}
\hypertarget{Toc484614211}{}{\selectlanguage{spanish}
Se utilizar\'an modelos 3D y texturas fotorealistas. No es necesario un gran nivel de detalle y/o de realismo pero se
evitar\'a una apariencia de estilo cartoon, comic o cel shading.}

\subsection[Influencias]{\selectlanguage{spanish} Influencias}
\hypertarget{Toc484614212}{}\subsubsection[Monkey island]{\selectlanguage{spanish} Monkey island}
\hypertarget{Toc484614213}{}{\selectlanguage{spanish}
Monkey island es uno de los referentes en cuanto a aventuras gr\'aficas se refiere.\ Da Vinci startup est\'a inspirado
en este juego y utiliza gran parte de sus mec\'anicas. Por ejemplo los di\'alogos interactivos en formato \'arbol, en
los que las decisiones tomadas por el jugador condicionan las respuestas de los personajes del juego.}

\subsubsection[Deadlight]{\selectlanguage{spanish} Deadlight}
\hypertarget{Toc484614214}{}{\selectlanguage{spanish}
Este juego posee un estilo similar al deseado en Da Vinci startup: un entorno 3D en el que el jugador se mueve en un
plano 2D.}

 \includegraphics[width=5.30458in,height=2.98414in]{anexos/GDD/GDD-img001.jpg} 

\section[Men\'us]{\selectlanguage{spanish} Men\'us}
\hypertarget{Toc484614215}{}\subsection[Men\'u principal]{\selectlanguage{spanish} Men\'u principal}
\hypertarget{Toc484614216}{}{\selectlanguage{spanish}
Es el men\'u que aparece inmediatamente despu\'es de que se abra el juego y el logo ``Made with Unity'' aparezca. En
este men\'u se pueden acceder a todas las opciones disponibles del juego. La ventana que aparece inicialmente es la que
contiene el t\'itulo ``Da Vinci startup'' y para navegar hacia el resto se debe de arrastrar la pantalla hacia derecha
o izquierda.\ }

{\selectlanguage{spanish}
Si se clica en el bot\'on play que hay en la pantalla inicial se empezar\'a el juego.}


\bigskip

{\selectlanguage{spanish}
 \includegraphics[width=5.90556in,height=1.73542in]{anexos/GDD/GDD-img002.png} \foreignlanguage{spanish}{\ \ }}

\subsection[Logros]{\selectlanguage{spanish} Logros}
\hypertarget{Toc484614217}{}{\selectlanguage{spanish}
Muestra los logros que han sido obtenidos durante el juego. En una l\'inea\ se muestra cu\'antos logros se han
conseguido y el total de los mismos.\ }

{\selectlanguage{spanish}
Bajo esta l\'inea se pueden ver las im\'agenes de los logros junto con un texto descriptivo. Los logros no completados
mostrar\'an su imagen en blanco y negro, mientras que los s\'i completados la mostrar\'an a color.\ }

{\selectlanguage{spanish}
A ambos lados de la\ hilera de im\'agenes se dispondr\'an de sendos botones con forma de flecha que al ser clicados
cambiar\'an el logro que se est\'a seleccionando en ese momento.}

{\selectlanguage{spanish}
El logro seleccionado se muestra con un tama\~no mayor. El texto descriptivo que aparece bajo la hilera de im\'agenes es
el correspondiente al logro seleccionado.\ }

 \includegraphics[width=1.35737in,height=2.49485in]{anexos/GDD/GDD-img003.png} 

\subsection[Cr\'editos]{\selectlanguage{spanish} Cr\'editos}
\hypertarget{Toc484614218}{}{\selectlanguage{spanish}
En un cuadro de texto se da cr\'edito a las personas necesarias: autores de scripts, im\'agenes, m\'usica u otros
elementos utilizados.}

{\selectlanguage{spanish}
Se a\~naden tambi\'en m\'etodos de contacto con el autor.}

 \includegraphics[width=1.30933in,height=2.548in]{anexos/GDD/GDD-img004.png} 

\subsection[Opciones]{\selectlanguage{spanish} Opciones}
\hypertarget{Toc484614219}{}{\selectlanguage{spanish}
En este men\'u se pueden cambiar diferentes\ aspectos del juegos tales como el volumen de la\ m\'usica o de los efectos
sonoros o\ eliminar los logros obtenidos.}

 \includegraphics[width=1.3923in,height=2.72745in]{anexos/GDD/GDD-img005.png} 

\subsection[Otros juegos]{\selectlanguage{spanish} Otros juegos}
\hypertarget{Toc484614220}{}{\selectlanguage{spanish}
Men\'u donde se pueden ver otros juegos creados por el autor. Al clicar en ellos el jugador es redirigido a\ la p\'agina
correspondiente en\ Google Play.}

 \includegraphics[width=1.38953in,height=2.73205in]{anexos/GDD/GDD-img006.png} 

\subsection[Juego]{\selectlanguage{spanish} Juego}
\hypertarget{Toc484614221}{}{\selectlanguage{spanish}
Durante el juego se dispone de una interfaz con\ 3\ botones que\ son el Indicador de personaje, el Indicador de
objetivos y el bot\'on de pausa.}

{\selectlanguage{spanish}
El indicador de personaje muestra una imagen del personaje (de los tres personajes controlables) que controla
actualmente el jugador. Al clicar en el Indicador de personaje este se expande, y al hacerlo se muestran las caras de
los tres personajes controlables. La cara del personaje que el jugador controla en ese momento se mostrar\'a en gris.}

{\selectlanguage{spanish}
Clicando en el icono de alguno de los otros jugadores har\'a que el men\'u se contraiga de nuevo y el personaje
controlado cambie al seleccionado.}

{\selectlanguage{spanish}
El Indicador de objetivos\ muestra el objetivo actual\ que se debe cumplir. Al clicar en el icono del Indicador de
objetivos se muestra una lista desplegable el\ texto descriptivo del objetivo.}

{\selectlanguage{spanish}
Al clicar en el bot\'on de pausa se expandir\'a dicho men\'u mostrando las diferentes opciones que ofrece y se parar\'a
el juego.}

 \includegraphics[width=4.08763in,height=3.53967in]{anexos/GDD/GDD-img007.png} 

\subsection[Men\'u conversacional]{\selectlanguage{spanish} Men\'u conversacional}
\hypertarget{Toc484614222}{}{\selectlanguage{spanish}
Este men\'u se abre cuando el jugador interact\'ua con un personaje. La respuesta del personaje\ con el que se est\'a
interactuando\ se muestra en el texto grande, mientras que las posibles respuestas que puede dar el jugador se muestran
en los botones.\ }

{\selectlanguage{spanish}
Cuando se da una respuesta con uno de los botones se actualiza la respuesta del personaje interactuado y\ el texto de
los botones.\ \ }

 \includegraphics[width=3.36728in,height=1.77154in]{anexos/GDD/GDD-img008.png} 

\subsection[Pausa]{\selectlanguage{spanish} Pausa}
\hypertarget{Toc484614223}{}{\selectlanguage{spanish}
Al clicar\ en el bot\'on de pausa situado en la esquina inferior derecha el juego se pausar\'a. Tambi\'en se abrir\'a un
men\'u desplegable en el que el jugador podr\'a seleccionar entre volver al men\'u principal o abrir el men\'u de
opciones.}

{\selectlanguage{spanish}
Clicando nuevamente en el bot\'on de pausa se cerrar\'a el desplegable y el juego se reactivar\'a.}

 \includegraphics[width=3.36029in,height=1.72756in]{anexos/GDD/GDD-img009.png} 

\section[Personajes]{\selectlanguage{spanish} Personajes}
\hypertarget{Toc484614224}{}\subsection[Controlables]{\selectlanguage{spanish} Controlables}
\hypertarget{Toc484614225}{}{\selectlanguage{spanish}
Cada uno de los personajes controlables tiene una fortaleza y debilidad en forma de NPCs con los que pueden negociar.
Durante el juego ser\'a necesario controlarlos a todos para poder negociar con los NPCs necesarios para avanzar en la
historia.}

\subsubsection[Leonardo da Vinci]{\selectlanguage{spanish} Leonardo da Vinci}
\hypertarget{Toc484614226}{}{\selectlanguage{spanish}
Protagonista de la historia. Se encarga de dialogar con Andrea del Verrocchio y es el ingeniero del taller. Es respetado
en la Logia de ingenieros por lo que le escuchar\'an cuando vaya. No puede entrar en el puerto y los burgueses no
negocian con \'el.}

\subsubsection[Luca Pacioli]{\selectlanguage{spanish} Luca Pacioli}
\hypertarget{Toc484614227}{}{\selectlanguage{spanish}
Tiene un aspecto desgarbado.\ Es\ uno de los ayudantes de Leonardo durante el juego.\ Puede entrar en el puerto y
negociar con los capitanes y marineros pero no puede negociar con los burgueses\ ni con los ingenieros de\ la Logia.}

\subsubsection[Salai]{\selectlanguage{spanish} Salai}
\hypertarget{Toc484614228}{}{\selectlanguage{spanish}
Luce un\ aspecto elegante. Los burgueses negocian con \'el pero\ no puede entrar en el puerto ni negociar con los
ingenieros de la Logia.}

\subsection[No controlables]{\selectlanguage{spanish} No controlables}
\hypertarget{Toc484614229}{}\subsubsection[Desarrollador]{\selectlanguage{spanish} Desarrollador}
\hypertarget{Toc484614230}{}{\selectlanguage{spanish}
Ataviado con ropas y complementos del siglo XXI aparece en diversos lugares a lo largo del juego. Al hablar con \'el te
pide que punt\'ues la app en Google Play, entre otras cosas.}

\subsubsection[Andrea del Verrocchio]{\selectlanguage{spanish} Andrea del Verrocchio}
\hypertarget{Toc484614231}{}{\selectlanguage{spanish}
Mentor de Leonardo. Da consejos sobre la metodolog\'ia Lean startup. Se encuentra en el taller.}

\subsubsection[Marinero\ portero]{\selectlanguage{spanish} Marinero\ portero}
\hypertarget{Toc484614232}{}{\selectlanguage{spanish}
Se encuentra en la entrada del puerto. Act\'ua como portero, dejando pasar solo a quien considera oportuno.\ Solo
dejar\'a pasar a Luca al puerto.}

\subsubsection[Marinero machaca 1]{\selectlanguage{spanish} Marinero machaca 1}
\hypertarget{Toc484614233}{}{\selectlanguage{spanish}
Puede ser contratado para trabajar en el taller de Leonardo.\ Se encuentra dentro del puerto.}

\subsubsection[Marinero machaca 2]{\selectlanguage{spanish} Marinero machaca 2}
\hypertarget{Toc484614234}{}{\selectlanguage{spanish}
Puede ser contratado para trabajar en el taller de Leonardo. Se encuentra dentro del puerto.}

\subsubsection[Capit\'an 1]{\selectlanguage{spanish} Capit\'an 1}
\hypertarget{Toc484614235}{}{\selectlanguage{spanish}
No quiere comprar el invento pero est\'a ansioso por venderte un barco. Est\'a en el puerto}

\subsubsection[Capit\'an 2]{\selectlanguage{spanish} Capit\'an 2}
\hypertarget{Toc484614236}{}{\selectlanguage{spanish}
No compra el invento pero te cuenta historias fantasiosas\ sobre sus viajes en el mar. Est\'a en el puerto}

\subsubsection[Capit\'an loco]{\selectlanguage{spanish} Capit\'an loco}
\hypertarget{Toc484614237}{}{\selectlanguage{spanish}
Rechaza comprar el invento de Leonardo en primera instancia. Acepta comprarlo despu\'es de que pivote hacia algo que le
interese m\'as (barco propulsado por h\'elice).\ Se encuentra dentro del puerto.}

\subsubsection[Burgu\'es 1]{\selectlanguage{spanish} Burgu\'es 1}
\hypertarget{Toc484614238}{}{\selectlanguage{spanish}
Acepta comprar el producto despu\'es de que lo hagan los capitanes, pero solo si se le a\~nade una gr\'ua para cargar
mercanc\'ias.\ Est\'a en el palacio.}

\subsubsection[Burgu\'es 2]{\selectlanguage{spanish} Burgu\'es 2}
\hypertarget{Toc484614239}{}{\selectlanguage{spanish}
No le interesa el invento pero te vende bonos del estado a un precio baj\'isimo. Est\'a en el palacio.}

\subsubsection[Burgu\'es 3]{\selectlanguage{spanish} Burgu\'es 3}
\hypertarget{Toc484614240}{}{\selectlanguage{spanish}
No le interesa el invento pero te aconseja sobre apuestas deportivas. Est\'a en el palacio.}

\subsubsection[Burgu\'es 4]{\selectlanguage{spanish} Burgu\'es 4}
\hypertarget{Toc484614241}{}{\selectlanguage{spanish}
No le interesa el invento pero te cuenta historias sobre sus a\~nos de gloria cuando era rico. Est\'a en el palacio.}

\subsubsection[Ingeniero 1]{\selectlanguage{spanish} Ingeniero 1}
\hypertarget{Toc484614242}{}{\selectlanguage{spanish}
Es el Gran Maestre.\ }

\subsubsection[Ingeniero 1]{\selectlanguage{spanish} Ingeniero 1}
\hypertarget{Toc484614243}{}{\selectlanguage{spanish}
Es un aprendiz de ingeniero.\ Se unir\'a el equipo de Leonardo si se le pide. Est\'a en la logia.}

\subsubsection[Ingeniero 1]{\selectlanguage{spanish} Ingeniero 1}
\hypertarget{Toc484614244}{}{\selectlanguage{spanish}
Es un ingeniero muy cre\'ido. Solo le interesa hablar de lo muy bueno que es y de los t\'itulos que tiene. Est\'a en la
logia.}

\subsubsection[Ciudadano\ de Florencia 1, 2]{\selectlanguage{spanish} Ciudadano\ de Florencia 1, 2}
\hypertarget{Toc484614245}{}{\selectlanguage{spanish}
Simplemente saluda.\ }

\subsubsection[Ciudadano de Florencia\ 3]{\selectlanguage{spanish} Ciudadano de Florencia\ 3}
\hypertarget{Toc484614246}{}{\selectlanguage{spanish}
Vende opio si le dices la contrase\~na correcta.}

\subsubsection[Ciudadano\ de Florencia 4]{\selectlanguage{spanish} Ciudadano\ de Florencia 4}
\hypertarget{Toc484614247}{}{\selectlanguage{spanish}
Pide dinero}

\subsubsection[Ciudadano\ de Florencia 5]{\selectlanguage{spanish} Ciudadano\ de Florencia 5}
\hypertarget{Toc484614248}{}{\selectlanguage{spanish}
\foreignlanguage{spanish}{Quiere que dones dinero para una asociaci\'on que suena muy falsa.}}

\section[Mec\'anicas de juego]{\selectlanguage{spanish} Mec\'anicas de juego}
\hypertarget{Toc484614249}{}\subsection[Movimiento]{\selectlanguage{spanish} Movimiento}
\hypertarget{Toc484614250}{}{\selectlanguage{spanish}
Los personajes controlables se pueden mover en el plano 2D. Para moverse, se\ toca\ en alg\'un lugar de la pantalla y el
personaje se mover\'a hacia\ la posici\'on tocada\ (solo en el eje x).}

\subsection[Interacci\'on con personajes]{\selectlanguage{spanish} Interacci\'on con personajes}
\hypertarget{Toc484614251}{}{\selectlanguage{spanish}
Al hacer\ tocar en la pantalla\ sobre un personaje el jugador se mover\'a hacia el personaje y al estar a su lado se
abrir\'a el men\'u de conversaci\'on.}

\subsection[Seleccionar personaje]{\selectlanguage{spanish} Seleccionar personaje}
\hypertarget{Toc484614252}{}{\selectlanguage{spanish}
Durante el juego se pueden controlar tres diferentes personajes: Leonardo, Luca y Salai. \ En cualquier momento (siempre
que no se est\'e en una conversaci\'on) se puede cambiar el personaje controlado usando el men\'u
correspondiente.\ Para ello se deber\'a expandir el selector de personaje y clicar en el icono de la cara del personaje
deseado.}

\subsection[Conversar]{\selectlanguage{spanish} Conversar}
\hypertarget{Toc484614253}{}{\selectlanguage{spanish}
Cuando se interact\'ua con un personaje se inicia una conversaci\'on. Para ello se abre autom\'aticamente el men\'u
conversaci\'on, que consiste en: se hace zoom sobre el juego hasta que los cuerpos de ambos personajes quedan en primer
plano;\ se crea una ventana de texto en la parte inferior donde aparece el discurso del personaje que nos est\'e
hablando; debajo del cuadro de texto aparecen botones con las posibles respuestas. Cuando se selecciona una respuesta
se actualiza el cuadro de texto con una contestaci\'on y se actualizan tambi\'en los botones de respuesta.}

\section[Mec\'anicas del mundo]{\selectlanguage{spanish} Mec\'anicas del mundo}
\hypertarget{Toc484614254}{}\subsection[Cambio de escenario]{\selectlanguage{spanish} Cambio de escenario}
\hypertarget{Toc484614255}{}{\selectlanguage{spanish}
Cada escenario tiene puertas que lo conecta con los escenarios adyacentes. Por ejemplo, desde el Taller de Leonardo se
puede acceder a las Calles de Florencia. A su vez desde las Calles de Florencia se puede acceder al puerto. El jugador
solo puede estar en un escenario a la vez, y clicando en las puertas mencionadas anteriormente se cambia de un
escenario a otro y\ se aparece en una posici\'on concreta del escenario (un punto de spawn situado al lado de la
puerta).}

\subsection[Scrolling paralax]{\selectlanguage{spanish} Scrolling paralax}
\hypertarget{Toc484614256}{}{\selectlanguage{spanish}
\foreignlanguage{spanish}{Cuando el jugador se mueve por el mapa el fondo se desplaza tambi\'en de forma horizontal. El
fondo se divide en varias capas, cada una representando un plano a diferente distancia del jugador, y estas se
desplazar\'an a diferente velocidad. Algunas de las capas se sit\'uan entre el jugador y la c\'amara, de forma que
tapan la visi\'on del jugador en ocasiones.}}

\subsection[Completar objetivo]{\selectlanguage{spanish} Completar objetivo}
\hypertarget{Toc484614257}{}{\selectlanguage{spanish}
Tras realizar la acci\'on asociada al objetivo actual, el icono del indicador de objetivo parpadear\'a y variar\'a
ligeramente de tama\~no de forma intermitente. Cuando el jugador abra el men\'u del indicador del objetivo aparecer\'a
un nuevo objetivo y el anterior ser\'a eliminado.}

\section[Escenarios]{\selectlanguage{spanish} Escenarios}
\hypertarget{Toc484614258}{}\subsection[Taller de Leonardo]{\selectlanguage{spanish} Taller de Leonardo}
\hypertarget{Toc484614259}{}{\selectlanguage{spanish}
En el taller se puede encontrar a Andrea del Verrocchio, que ser\'a quien dicte los objetivos a cumplir durante la
historia y dar\'a consejos sobre Lean startup.}

{\selectlanguage{spanish}
Conecta con las Calles de Florencia.}

\subsection[Calles de Florencia]{\selectlanguage{spanish} Calles de Florencia}
\hypertarget{Toc484614260}{}{\selectlanguage{spanish}
Las calles de Florencia es el escenario conector en el juego: desde las calles se puede acceder al resto de escenarios
como el taller, el puerto, la logia o el palacio.}

{\selectlanguage{spanish}
Los ciudadanos de Florencia pululan por las calles y\ el jugador podr\'a hablar con ellos.}

\subsection[Palacio de Lorenzo]{\selectlanguage{spanish} Palacio de Lorenzo}
\hypertarget{Toc484614261}{}{\selectlanguage{spanish}
En este edificio se encuentran los burgueses, que solo har\'an negocios con Salai.}

{\selectlanguage{spanish}
Se accede desde las calles de Florencia.}

\subsection[Logia de ingenieros]{\selectlanguage{spanish} Logia de ingenieros}
\hypertarget{Toc484614262}{}{\selectlanguage{spanish}
En este edificio se encuentran los ingenieros, que solo hablar\'an con Leonardo. Se accede desde las calles de
Florencia.}

\subsection[Puerto]{\selectlanguage{spanish} Puerto}
\hypertarget{Toc484614263}{}{\selectlanguage{spanish}
Se encuentran marinos rudos que\ est\'an ociosos. Un marino en la entrada del puerto act\'ua como portero\ dejando
entrar a quien\ considera oportuno. Salai no puede entrar debido a que por su aspecto de clase alta los marineros no le
dejan pasar. Leonardo no puede entrar ya que el puerto es un lugar peligroso y los marineros no le dejan pasar por si
le ocurre algo.}

{\selectlanguage{spanish}
Los marineros pueden ser contratados por Luca para trabajar en el taller.}

\section[Guion\ literario]{\selectlanguage{spanish} Guion\ literario}
\label{guionLiterario}
\hypertarget{Toc484614264}{}{\selectlanguage{spanish}
Solo los ingenieros m\'as sobresalientes son nombrados Gran maestre en la logia de ingenieros de Florencia. Esta es la
ambici\'on de Leonardo da Vinci, el m\'as grande de los ingenieros de Florencia. Ha construido multitud de artefactos
pero todav\'ia tiene un \'ultimo reto por delante antes de pasar de aprendiz a Gran maestre: debe construir un invento
que se venda por un mill\'on de florines.}

{\selectlanguage{spanish}
Leonardo\ est\'a contemplando el vuelo de un p\'ajaro en las calles de Florencia y queda fascinado por la facilidad con
la que la criatura se eleva hacia los cielos. Se le ocurre que si consiguiera construir un artefacto que pudiera hacer
volar a las personas de la misma forma que lo hacen las aves se har\'ia rico\ y se convertir\'ia en un Gran
Maestre.\ Llamar\'a\ a este invento\ {}``el vuelac\'optero''.}

{\selectlanguage{spanish}
Con esta idea en mente va corriendo al taller en el que trabaja junto con su maestro Andrea del Verrochio y sus
aprendices Salai y Luca Paccioli. Una vez all\'i le cuenta la idea a Verrochio y este le promete su ayuda a lo largo de
todo el proceso de construir y comercializar el invento puesto que es un gur\'u de una\ nueva metodolog\'ia de trabajo
inventada en ``Toscana Valley''.\ }

{\selectlanguage{spanish}
Leonardo est\'a ansioso por ponerse manos a la obra\ pero Verrochio le advierte de que para que un proyecto funcione no
se puede trabajar a lo cowboy, hace falta un equipo que aporte los conocimientos que uno no tiene.\ Es necesario
conocer gente, y eso solo se puede hacer saliendo a la calle. En Toscana Valley llaman a esto ``hacer networking''.\ De
esta forma la primera tarea del taller es encontrar a unos obreros que ayuden a construir. Estos obreros ser\'an
marineros parados del puerto, al que solo\ es permitido el acceso a Luca.}

{\selectlanguage{spanish}
Una vez conseguida la mano de obra es necesario conseguir financiaci\'on, para lo cual se pueden tomar diferentes
alternativas que Verrocchio explicar\'a: se puede optar por el bootstrapping o buscar financiaci\'on externa. El taller
deber\'a optar por la segunda opci\'on ya que con el capital que tienen no pueden asumir los costes del proyecto. Para
conseguir financiaci\'on el lugar m\'as adecuado es el palacio donde los burgueses se re\'unen a discutir\ sobre
econom\'ia. Solo Salai, con su estatus de burgu\'es, conseguir\'a que los burgueses le tomen en serio y le proporcionen
los fondos que necesitan.}

{\selectlanguage{spanish}
Conseguidos financiaci\'on y mano de obra, solo falta hacer ingenier\'ia y dise\~nar el producto que se va a construir.
Para ello Leonardo deber\'a convencer a alg\'un ingeniero de la logia para que se una al equipo, y una vez conseguido,
trabajar en los planos.}

{\selectlanguage{spanish}
Leonardo est\'a impaciente, con todos los recursos reunidos solo falta ponerse manos a la obra y construir el
vuelac\'optero. Una vez construido todo el mundo querr\'a uno y se har\'an ricos. Craso error. Verrocchio le quita la
idea r\'apidamente de la cabeza: as\'i no se hacen las cosas en Toscana Valley. Las ideas suenan perfectas en nuestra
cabeza pero rara vez coinciden con las de los\ clientes. As\'i que en lugar de construir el producto y esperar a que
est\'e finalizado para\ venderlo, lo que har\'an ser\'a utilizar un desarrollo iterativo: construir\'an una versi\'on
temprana del vuelac\'optero y le ir\'an a\~nadiendo partes a medida que los clientes digan lo que les gusta y lo que
no.}

{\selectlanguage{spanish}
Pasan unas semanas, la primera iteraci\'on del desarrollo ha sido completada y Leonardo va a hablar con Verrocchio para
preguntar cu\'ales son los siguientes pasos.\ Ahora que tienen un MVP (producto m\'inimo viable) es hora de salir a la
calle y conseguir lo que en Toscana Valley se llama ``early adopters'': gente tan loca como ellos que se atrevan a
comprar un producto novedoso a\'un sin finalizar.}

{\selectlanguage{spanish}
La b\'usqueda es infructuosa y al parecer nadie quiere comprar un vuelac\'optero, pero un capit\'an loco del puerto
sugiere que con ciertos cambios tal vez estar\'ia interesado.}

{\selectlanguage{spanish}
Leonardo le cuenta sobre el desastre a Verrocchio y este le explica que gracias a que no han estado durante meses
gastando dinero y construyendo una versi\'on definitiva del producto, ahora pueden reaccionar a tiempo y cambiar lo que
sea necesario para poder venderlo.}

{\selectlanguage{spanish}
Cuando hay que hacer grandes cambios sobre la hip\'otesis de negocio, se habla de pivotar. De forma que realizando estos
cambios la hip\'otesis de negocio evoluciona de acuerdo a la experiencia adquirida y se acerca m\'as a un modelo de
negocio viable.}

{\selectlanguage{spanish}
De acuerdo a lo requerido por el capit\'an loco, incorporar\'an el vuelac\'optero a un barco para que\ navegue m\'as
deprisa. Una vez construido el nuevo barco se le comunica al capit\'an loco y este lo compra muy gustosamente.}

{\selectlanguage{spanish}
Leonardo le da la noticia a Verrocchio y ambos lo celebran, pero tienen que seguir consiguiendo ventas as\'i que hay que
volver a la calle.\ Al hablar con los burgueses a estos les parece interesante el producto pero no se acomoda del todo
a sus necesidades: lo comprar\'an solo si dispone de una gr\'ua para cargar y descargar mercanc\'ias.}

{\selectlanguage{spanish}
Sabiendo esto, Verrocchio explica que el producto se tiene que adaptar a las necesidades y deseos de los clientes. Esto
se denomina ``customer development''. Los peque\~nos cambios en el modelo de negocio, \ como a\~nadir la gr\'ua al
barco, se denominan pivotar.}

{\selectlanguage{spanish}
Con los nuevos cambios en el barco los burgueses lo compran encantados y Leonardo consigue la meta que ten\'ia que
cumplir para convertirse en Gran maestre. Reflexionando con Verrocchio se da cuenta de c\'omo ha evolucionado la idea
que tuvo originalmente (el vuelac\'optero) hasta convertirse en el barco mercante propulsado por h\'elice. Tambi\'en
piensan en c\'omo habr\'ia resultado todo si no hubieran empleado una metodolog\'ia de desarrollo \'agil y hubieran
empleado muchos esfuerzos en construir el vuelac\'optero antes de mostrarlo a los potenciales clientes.\ }

{\selectlanguage{spanish}
Finalmente Leonardo sale del taller para ir a la logia de ingenieros y la puerta del taller est\'a abarrotada de gente
que quiere comprar uno de sus barcos. La fama de sus veloces barcos mercantes se ha extendido por toda Florencia. En la
logia es convertido en Gran maestre y el ingeniero jefe le explica que la b\'usqueda de su startup ha finalizado. Han
conseguido encontrar su hueco en el mercado y desarrollar su producto. Da Vinci\ startup deja de ser una startup.}

\section[Objetivos]{\selectlanguage{spanish} Objetivos}
\hypertarget{Toc484614265}{}{\selectlanguage{spanish}
El flujo del juego es controlado por el objetivo a cumplir. En todo momento se dispone de un objetivo a cumplir, y al
hacerlo se desbloquea el siguiente y el mundo del juego se actualiza.}

{\selectlanguage{spanish}
De este modo el transcurso del juego se basa en el cumplimiento de los objetivos y son el hilo conductor de la historia.
La lista de los objetivos ordenados por orden de aparici\'on es:}

\liststyleLFOxii
\setcounter{saveenum}{\value{enumi}}
\begin{enumerate}
\setcounter{enumi}{\value{saveenum}}
\item {\selectlanguage{spanish}
Habla con el Gran Maestre}
\item {\selectlanguage{spanish}
Ve al taller y pide ayuda al maestro Verrochio\ \ }
\item {\selectlanguage{spanish}
Ve al puerto y consigue constructores}
\item {\selectlanguage{spanish}
Ve al taller y habla con el maestro Verrochio}
\item {\selectlanguage{spanish}
Consigue financiaci\'on de los burgueses}
\item {\selectlanguage{spanish}
Ve al taller y habla con el maestro Verrocchio}
\item {\selectlanguage{spanish}
Consigue un ingeniero en la logia}
\item {\selectlanguage{spanish}
Ve al taller y habla con el maestro Verrocchio}
\item {\selectlanguage{spanish}
Busca early adopters}
\item {\selectlanguage{spanish}
Ve al taller y cuentale el fracaso al maestro Verrocchio}
\item {\selectlanguage{spanish}
Habla con el capit\'an loco}
\item {\selectlanguage{spanish}
Ve al taller y habla con el maestro Verrocchio}
\item {\selectlanguage{spanish}
Busca m\'as early adopters}
\item {\selectlanguage{spanish}
Ve al taller y habla con el maestro Verrocchio}
\item {\selectlanguage{spanish}
Habla con el burgu\'es}
\item {\selectlanguage{spanish}
Ve al taller y habla con el maestro Verrocchio}
\item {\selectlanguage{spanish}
Habla con el gran maestre}
\end{enumerate}
\section[Logros]{\selectlanguage{spanish} Logros}
\hypertarget{Toc484614266}{}{\selectlanguage{spanish}
Adem\'as de los objetivos se podr\'an desbloquear logros, aunque el desbloqueo de estos no es requerido para completar
el juego. Su funci\'on es meramente coleccionista. Los logros que se podr\'an desbloquear son:}

\liststyleLFOxi
\begin{itemize}
\item {\selectlanguage{spanish}
La ira de los bits caer\'a sobre ti}
\item {\selectlanguage{spanish}
Opio conseguido. Es hora de montar una buena fiesta}
\item {\selectlanguage{spanish}
Has conquistado el coraz\'on de un developer}
\item {\selectlanguage{spanish}
Has completado el primer objetivo}
\item {\selectlanguage{spanish}
Has completado el juego}
\end{itemize}
\section[Assets requeridos]{\selectlanguage{spanish} Assets requeridos}
\hypertarget{Toc484614267}{}\subsection[Modelos 3D]{\selectlanguage{spanish} Modelos 3D}
\hypertarget{Toc484614268}{}\subsubsection[Personajes]{\selectlanguage{spanish} Personajes}
\hypertarget{Toc484614269}{}{\selectlanguage{spanish}
Se necesitar\'a un modelo por cada uno de los personajes del juego. Algunos de ellos reutilizar\'an el mismo modelo
y\ solo\ cambiar\'an las texturas que utilizan.}

{\selectlanguage{spanish}
Los modelos requeridos son:}

\liststyleLFOvi
%\setcounter{saveenum}{\value{enumi}}
\begin{itemize}
%\setcounter{enumi}{\value{saveenum}}
\item {\selectlanguage{spanish}
Leonardo}
\item {\selectlanguage{spanish}
Luca}
\item {\selectlanguage{spanish}
Salai}
\item {\selectlanguage{spanish}
Verrocchio}
\item {\selectlanguage{spanish}
Capit\'an (compartido por los dos capitanes y el capit\'an loco)}
\item {\selectlanguage{spanish}
Marinero (compartido por los dos marineros)}
\item {\selectlanguage{spanish}
Burgu\'es (compartido por todos los burgueses)}
\item {\selectlanguage{spanish}
Ingeniero (compartido por todos los ingenieros)}
\item {\selectlanguage{spanish}
Desarrollador}
\end{itemize}
\subsubsection[Edificios]{\selectlanguage{spanish} Edificios}
\hypertarget{Toc484614270}{}{\selectlanguage{spanish}
Se requerir\'an edificios para todos los escenarios del juego. En algunos de los escenarios (taller, puerto) se
necesitar\'a utiler\'ia que represente el trabajo que ah\'i se hace.}

\subsubsection[M\'usica]{\selectlanguage{spanish} M\'usica}
\hypertarget{Toc484614271}{}{\selectlanguage{spanish}
Se requerir\'a una pista de sonido ambiente por cada uno de los escenarios:}

\liststyleLFOvii
\setcounter{saveenum}{\value{enumi}}
\begin{itemize}
\setcounter{enumi}{\value{saveenum}}
\item {\selectlanguage{spanish}
Taller de Leonardo}
\item {\selectlanguage{spanish}
Logia de ingenieros}
\item {\selectlanguage{spanish}
Calles de Florencia}
\item {\selectlanguage{spanish}
Palacio de Lorenzo}
\item {\selectlanguage{spanish}
Puerto}
\end{itemize}
{\selectlanguage{spanish}
Adem\'as se incluir\'a una pista de sonido para el men\'u principal.}

\subsection[Sprites]{\selectlanguage{spanish} Sprites}
\hypertarget{Toc484614272}{}\subsubsection[IU]{\selectlanguage{spanish} IU}
\hypertarget{Toc484614273}{}{\selectlanguage{spanish}
Durante el juego:}

\liststyleLFOxiii
\begin{itemize}
\item {\selectlanguage{spanish}
Icono selector personajes\ }

\begin{itemize}
\item {\selectlanguage{spanish}
Icono cara Salai}
\item {\selectlanguage{spanish}
Icono cara Luca}
\item {\selectlanguage{spanish}
Icono cara Leonardo}
\end{itemize}
\item {\selectlanguage{spanish}
Icono indicador de objetivos}
\item {\selectlanguage{spanish}
Icono pausa}

\begin{itemize}
\item {\selectlanguage{spanish}
Icono home}
\item {\selectlanguage{spanish}
Icono men\'u opciones}
\end{itemize}
\end{itemize}
{\selectlanguage{spanish}
En el men\'u principal:}

\liststyleLFOxiv
\begin{itemize}
\item {\selectlanguage{spanish}
Fondo de los diferentes men\'us}
\item {\selectlanguage{spanish}
Bot\'on iniciar juego}
\item {\selectlanguage{spanish}
Iconos de logros}
\item {\selectlanguage{spanish}
Flechas de navegaci\'on entre logros}
\item {\selectlanguage{spanish}
Marco para texto de logros}
\item {\selectlanguage{spanish}
Icono Linkedin}
\item {\selectlanguage{spanish}
Iconos altavoces con diferentes vol\'umenes}
\end{itemize}

\bigskip


\bigskip


\bigskip
%\end{document}

%\input{glosario/entradas_glosario}
% \addcontentsline{toc}{chapter}{Glosario} %si se usa glosario hay que añadirlo al índice
% \printglossary %muestra el glosario

\end{document}
