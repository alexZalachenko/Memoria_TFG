\chapter{Marco Teórico}
\label{marcoteorico}

Como apoyo a los cursos formativos sobre emprendimiento han surgido multitud de juegos para enseñar y aplicar de forma práctica conceptos sobre la creación de iniciativas empresariales.\\
Estos juegos son una parte importante de la formación ya que permiten aprender de forma más amena. Además el utilizar de forma práctica los conocimientos adquiridos ayuda a que se entiendan mejor y se recuerden durante más tiempo.
Para los juegos mencionados anteriormente existen varios formatos que serán explicados en detalle a continuación. 

\section{Actividades en grupo}

Es el tipo más sencillo y tradicional. Solo necesita de los participantes y una actividad previamente elegida. Una de las ventajas que tiene este tipo de actividades es que ponen a las personas en contacto directo, de modo que tienen que dejar a un lado la verguenza e interactuar como lo harían ante clientes, inversores o trabajadores. Además estos encuentros pueden servir para hacer contactos útiles en un futuro.\\
Este tipo de actividades potencian las habilidades sociales y la creatividad de los participantes ya que los únicos elementos del juego son las personas, y las mecánicas del juego son sus discursos, explicaciones, gestos, actuaciones, etc.

\section{Juegos de mesa}

Los juegos de mesa mantienen muchos de los aspectos positivos de las actividades en grupo ya que también son presenciales, con las ventajas y desventajas que ello conlleva.\\
Además estos juegos pueden ser más divertidos debido a que introducen elementos como tableros, cartas, textos o ilustraciones entre otros elementos. Son especialmente interesantes para personas que no se sientan cómodas con las asctividades en grupo debido al alto grado de interacción que demandan.

También es más fácil el uso de mecánicas complejas ya que hay instrumentos para contabilizar y describir el estado del juego: dados para contar el número de vidas, poder mágico o turnos; tableros con diferentes casillas, territorios o zonas; fichas de jugador con parámetros, habilidades, y características.

Actualmente hay numerosos juegos de mesa entre los que destacaremos:
\begin{itemize}
         \item Colonos de Catán: \\

Apto para 6 jugadores (a través de una expansión), este juego de gestión de recursos y comercio nos pone en la piel de un colono que debe ir construyendo sus aldeas y caminos. En Colonos de Catán prima tu habilidad para negociar por el contrario y tu capacidad de estrategia a medio y largo plazo. \cite{faceentrepreneurship2016}

El Catán [...] evita el enfrentamiento tan directo y obliga a negociar para ganar.

Ahí reside una de las claves, en el hecho que de entrada nadie posea recursos suficientes de todos los tipos para progresar. En cada turno se comercia con las materias primas, un trueque básico que permite saber cómo funciona un mercado libre en el que cada uno tiene sus propios intereses. \cite{albertini2015}

Lleva unas 18 millones de copias vendidas [...] Ha aparecido en 'The Big Bang Theory' [...] Mark Zuckerberg se ha declarado adicto, es uno de esos juegos a los que debes jugar si no quieres ser el ''margi" de Silicon Valley \cite{albertini2015}

         \item Pandemia: 
En Pandemia somos un grupo de hasta 4 científicos que tienen que mantener a raya una serie de virus, o de lo contrario la Humanidad tendrá un grave problema. [...] debes aprender a formar equipo y hacerlo funcionar si quieres tener éxito en tu startup. En este sentido, Pandemia puede ser la terapia perfecta para ti y tu equipo ya que o colaboráis y funcionáis perfectamente engrasados o correréis el riesgo de fracasar. Y también en la vida real. \cite{faceentrepreneurship2016}

         \item Flea market:
juego de dados en el que tendrás que descubrir los tesoros escondidos en un mercado de segunda mano. Tú serás el cliente que compra, y tu objetivo es adquirir bienes lo más barato posible para venderlos por más dinero una vez el dado de la demanda dice que ya son populares de nuevo. \cite{mariagonzalez2015}
\end{itemize}


\section{Videojuegos}
