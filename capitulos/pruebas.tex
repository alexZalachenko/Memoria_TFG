\chapter{Pruebas y validación}

Para el desarrollo no se ha optado por la escritura de tests automáticos ya que el tiempo requerido para este proceso habría sido inasumible. La utilización de tests automáticos es un proceso imprescindible en la ingeniería de software pero lamentablemente para este proyecto no ha sido posible.

Sin embargo sí que se han ejecutado diversas pruebas para comprobar que el sistema se comporta de forma adecuada y cumple con unos criterios de calidad.

\section{Pruebas exploratorias}

El la forma de testing más sencilla. Jugar el juego haciendo especial incapié en la búsqueda de bugs o errores. De esta forma se han encontrado multitud de comportamientos no deseados. Ha sido una de las formas de testing más utilizadas en este proyecto.

\section{Pruebas funcionales}

Haciendo uso de los requerimientos especificados en el apartado \nameref{requisitosUsuario} se ha comprobado que estaban correctamente implementados mediante la interacción directa con la aplicación final. De esta forma se validan dichos requisitos ya que se prueban tal cual lo haría un jugador.
Se han validado todos los requisitos implementados y se ha comprobado que funcionan de forma correcta.

\section{Pruebas de usabilidad}

Haciendo uso de un grupo de voluntarios que deseaban probar el juego se les dejó jugar durante un tiempo para luego realizarles una serie de preguntas. 
Se les pidió que valoraran la sencillez con la que se entendía las funcionalidades que provee la interfaz. Se les pidió también que valoraran la facilidad con la que el juego indicaba al jugador lo que debía hacer.
La última batería de pruebas fue pedir a los usuarios que realizaran tareas sencillas tales como cambiar de personaje o cambiar de escenario. El aspecto a evaluar era el tiempo que tardaban en completar la tarea y cuanto les había costado descubrir como realizarla.


En general el resultado de las pruebas fue satisfactorio. Las interfaces tienen el mínimo de elementos posibles y las mecánicas de juego tampoco son muy elevadas por lo tanto es sencillo desenvolverse con el sistema.

\section{Pruebas de compatibilidad}

El sistema se probó en diferentes dispositivos con tamaños de pantalla y resoluciones variadas. El objetivo era verificar que las proporciones del juego, la estructura de la interfaz y de los escenarios no se veian afectadas por la resolución del dispositivo. Los dispositivos probados fueron un Xiaomi redmi3 con el sistema operativo Android y una resolución de pantalla de 1920x1080 en 5.5 pulgadas y una tableta Nexus 7 con el sistema operativo Android y una resolución de 1920x1200 en 7 pulgadas


En ambos dispositivos los resultados fueron los esperados: la interfaz se adapta a la pantalla perfectamente. 


