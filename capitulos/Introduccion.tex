\chapter{Introducción}

El 12 de octubre de 1492 un temerario explorador, Cristobal Colón, y su tripulación pisan la arena de una isla muy al oeste de Europa conocida como Guanahani. Este hecho marca un hito en la historia de la humanidad pues los cambios culturales, económicos, políticos y militares que produce dan lugar a la llamada Edad Moderna.\\
Colón vio una oportunidad de negocio en el control de las rutas comerciales que unían Europa con Asia pues eran recorridas por miles de comerciantes que traían especias y productos de lujo desde las tierras de Extremo Oriente. El comercio además se realizaba por tierra, lo que lo convertía en un proceso lento, inseguro e ineficiente, además de enriquecedor para los árabes que controlaban las rutas comerciales.

El proyecto tenía un gran interés económico pues como se ha dicho anteriormente, el control de una ruta comercial con Asia era muy lucrativo, pero a su vez tenía un gran riesgo ya que el futuro de la expedición era tremendamente incierto y había pocas posibilidades de encomendarse al vasto oceano y volver para contarlo. Debido a esta incertidumbre sobre el retorno de la inversión a Colón le fue complicado encontrar financiación para su proyecto, hasta que finalmente, tras recurrir a varios monarcas y mecenas,  los Reyes Católicos le proveyeron de los rescursos necesarios para iniciar su aventura.

Se podría considerar a Cristobal Colón como un emprendedor, a pesar de que el término fue usado por primera vez doscientos años después por el economista Richard Cantillon que define al emprendedor como ''La persona que paga un cierto precio para revender un producto a un precio incierto, por ende tomando decisiones acerca de la obtención y el uso de recursos, y admitiendo consecuentemente el riesgo en el emprendimiento". De esta definición se puede apreciar que un emprendedor inicia proyectos y acepta la incertidumbre y el riesgo que ello conlleva, puesto que en caso de desastre es él quien carga con la responsabilidad.

La actitud emprendedora ha sido una constante a lo largo de la historia de la humanidad: desde Cristobal Colón hasta Bill Gates, pasando por Leonardo Da Vinci, Henry Ford o Nikola Tesla; hombres y mujeres con coraje han empezado proyectos bajo una idea prometedora y asumiendo grandes riesgos, motivados por la pasión y las perspectivas de éxito. 

El emprendimiento es una actividad especialmente necesaria para el progreso de una sociedad pues es un proceso que crea riqueza, innovación y empleo. Los emprendedores crean productos y servicios revolucionarios que hacen la vida de las personas más fácil, mejorando por ello su calidad de vida. Además suele ser una salida muy recurrida en épocas de crisis económicas debido a la escasez de empleo.

\section{Emprendimiento y el fenómeno startup}

Cada vez es más frecuente escuchar el término startup, pequeñas empresas dedicadas al ámbito tecnológico que alcanzan en pocos años grandes cuotas de mercado y se venden por millones de euros a empresas más grandes.

 El fenómeno goza de tanta popularidad que ha inspirado incluso a series como Silicon Valley, que narra las aventuras de un grupo de jóvenes ingenieros que crean una startup tecnológica y se enfrentan al reto de sobrevivir en un ecosistema hostil como es el mercado; la película Piratas de Silicon Valley, que narra la historia de enfrentamiento entre Microsoft y Apple; la película La red social que cuenta la historia de Mark Zuckerberg y como crea la red social Facebook.

Llegado a este punto cabe preguntarse: ¿Qué es exactamente una startup?. Es un error común pensar que las startup son simplemente versiones más pequeñas de empresas grandes. En palabras de los gurús del emprendimiento Steve Blank y Bob Dorf, una startup es ''una organización temporal en busca de un modelo de negocio rentable, que pueda repetirse y que es escalable".\\
De la anterior definición se puede extraer que una startup:
\begin{itemize}
	\item Es una organización temporal, es decir, el objetivo no es ser siempre una startup. El objetivo es convertirse en una empresa consolidada.
	\item No conoce con seguridad cual va a ser su actividad. En su lugar parten de un modelo de negocio temporal que va evolucionando a medida que interactúa con el mercado.
	\item Busca un modelo de negocio repetible y escalable, que le permita ejecutar dicho modelo de negocio durante un tiempo indefinido y además expandirse.
\end{itemize}
El emprendimiento es inherente al fenómeno startup pues la incertidumbre es un pilar fundamental al crear una de estas empresas, que ni siquiera tienen un modelo de negocio que se pueda asegurar que va a funcionar.

\section{Estado actual del emprendimiento en España}

Si bien el fenómeno startup nació en EEUU y es allí donde está más consolidado, en España es una tendencia igualmente extendida. Atendiendo a cifras de financiación En 2015, las startups españolas lograron financiación por valor de 500 millones de euros, un 87\%   más que en 2014, cuando apenas se invirtieron 286 millones de euros(citar http://www.ticbeat.com/entrevistas/asi-laten-startups-espana/).\\
Actualmente en nuestro país hay 1783 empresas emergentes \cite{startupxplore2017} distribuidas principalmente en Madrid, Cataluña y la Comunidad valenciana. Dichas empresas se dedican principalmente al ecommerce(22\%), social media(13\%) y las empresas(12\%). En cuanto a la financiación, 172 inversores 



paco 
\section{¡Importante!, leer primero}

Este texto está escrito pensando en orientar a los alumnos que usarán \LaTeX para escribir sus TFG de Ingeniería Multimedia.

Contiene información útil para aquellos que no tengan experiencia previa en \LaTeX así como algunos datos acerca de cómo escribir mejor su TFG.

A continuación, se ofrece una copia de la información que hay en el libro de estilo para la realización de los TFG de la EPS de la Universidad de Alicante.

En el siguiente capítulo (página \pageref{marcoteorico}) encontrarás algunos ejemplos de cómo hacer listas, tablas y otras estructuras de un texto en \LaTeX. Con un poco de paciencia, estudia cómo se hacen estas cosas y luego aplícalas en tus documentos.

\section{Estructura de un TFG}

En caso de que el TFG/TFM tenga como finalidad la elaboración de un proyecto o un 
informe científico o técnico, deberá ajustarse a lo dispuesto en las normas UNE 
157001:2002 y UNE 50135:1996 respectivamente.

Si el TFG/TFM tiene por finalidad la elaboración de un trabajo monográfico, el 
documento presentado deberá constar de las siguientes partes, teniendo como base la 
norma UNE 50136:1997.

\begin{description}
\item[Preámbulo:] se describirán brevemente la motivación que ha originado la realización del TFG/TFM, así como de una breve descripción de los objetivos generales que se quieren alcanzar con el trabajo presentado.
\item[Agradecimientos:] se podrá añadir las hojas necesarias para realizar los agradecimientos, a veces obligatorios, a las entidades y organismos colaboradores.
\item[Dedicatoria:] se podrá añadir una única hoja con dedicatorias, su alineación será derecha y centradas de forma distribuida en la página.
\item[Citas:] (frases célebres) se podrá añadir una única hoja con citas, su alineación será derecha y centradas de forma distribuida en la página.
\item[Índices:] cada índice debe comenzar en una nueva página, se incluirán los índices que se estimen necesarios (conforme UNE 50111:1989), en este orden:
\begin{description}
\item[Índice de contenidos:] (obligatorio siempre) se incluirá un índice de las secciones de las que se componga el documento, la numeración de las 
divisiones y subdivisiones utilizarán cifras arábigas (según UNE 50132:1994) y harán mención a la página del documento donde se ubiquen.
\item[Índice de figuras:] si el documento incluye figuras se podrá incluir también un índice con su relación, indicando la página donde se ubiquen.
\item[Índice de tablas:] en caso de existir en el texto, ídem que el anterior.
\item[Índice de abreviaturas, siglas, símbolos, etc.:] en caso de ser necesarios se podrá incluir cada uno de ellos.
\end{description}
\item[Cuerpo del documento:] en el contenido del documento se da flexibilidad para su organización y se puede estructurar en las secciones que se considere. En todo caso obligatoriamente se deberá, al menos, incluir los siguientes contenidos:
\begin{description}
\item[Introducción:] donde se hará énfasis a la importancia de la temática, su vigencia y actualidad; se planteará el problema a investigar, así como el propósito o finalidad de la investigación.
\item[Marco teórico o Estado del arte:] se hará mención a los elementos conceptuales que sirven de base para la investigación, estudios previos relacionados con el problema planteado, etc.
\item[Objetivos:] se establecerá el objetivo general y los específicos.
\item[Metodología:] se indicará el tipo o tipos de investigación, las técnicas y los procedimientos que serán utilizados para llevarla a cabo; se identificará la población y el tamaño de la muestra así como las técnicas e instrumentos de recolección de datos.
\item[Resultados:] incluirá los resultados de la investigación o trabajo, así como el análisis y la discusión de los mismos.
\end{description}
\item[Conclusiones:] obligatoriamente se incluirá una sección de conclusiones donde se realizará un resumen de los objetivos conseguidos así como de los resultados obtenidos si proceden.
\item[Bibliografía y referencias:] se incluirá también la relación de obras y materiales consultados y empleados en la elaboración de la memoria del TFG/TFM. La bibliografía y las referencias serán indexadas en orden alfabético (sistema nombre y fecha) o se numerará correlativamente según aparezca (sistema numérico). Se empleará la familia 1 como tipo de letra. Podrá utilizarse cualquier sistema bibliográfico normalizado predominante en la rama de conocimiento, estableciéndose como prioritarios el sistema ISO 690, sistema APA (American Psychological Association) o Harvard (no necesariamente en ese orden de preferencia). En esta plantilla Latex se propone usar el estilo APA indicándolo en la línea correspondiente como 
\begin{verbatim}
\bibliographystyle{apalike}
\end{verbatim}


\item[Anexos:] se podrá incluir los anexos que se consideren oportunos.

\end{description}



